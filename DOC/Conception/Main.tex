\documentclass[12pt,letterpaper]{article}
\usepackage{hyperref}
\hypersetup{colorlinks=true,allcolors=black}
\usepackage[utf8]{inputenc}
\usepackage[frenchb]{babel}
\usepackage[T1]{fontenc}
\usepackage{ae,aecompl}
\usepackage{pslatex}
\usepackage[top=2cm, bottom=2cm, left=2cm, right=2cm]{geometry}
\usepackage{hhline}
\usepackage{amsmath}
\usepackage{pgfplots}
\pgfplotsset{compat=newest}
\usepackage{multicol}
\usepackage{graphicx}
\graphicspath{{Images/}}
\hyphenpenalty=100
\usepackage{caption}
\usepackage{enumitem}
\usepackage{textcomp}
% Pour multicol (Align top)
\raggedcolumns
% Pour enlever l'indentation et avoir un espace entre les paragraphes
%\usepackage[parfill]{parskip}
\usepackage{float}
\floatstyle{plaintop}
\restylefloat{table}

\begin{document}
	
	%====== Page de presentation ======%
	\hypersetup{pageanchor=false}
	
%====== Page de presentation ======%

\title{Eclipse X - BMS}
\author{
	Daigneault-St-Arnaud, Christian, DAIC30099006
}
\newcommand{\cours}{ }
\newcommand{\prof}{ }

\makeatletter
\begin{titlepage}
	\begin{center}
	\pagenumbering{gobble}
	
	{\Huge \@title}\\ 
	\vspace{3cm}
	{\large Par \\
		\vspace{0.5cm}
		\@author \\
		\vspace{3cm}
		%\cours \\
		\vspace{0.5cm}
		%\prof \\
		\vspace{3.5cm}
		\@date \\
		\vspace{3.5cm}
		\'{E}COLE DE TECHNOLOGIE SUP\'{E}RIEURE \\
		UNIVERSIT\'{E} DU QUÉBEC
	}		
	\end{center}

	
\end{titlepage}
\makeatother




	\newpage
	%====== Table des matieres ======%
	\pagenumbering{roman}
	\hypersetup{pageanchor=true}
	\tableofcontents
	\listoftables
	\listoffigures	
	\newpage


%====================== INCLUSION DES PARTIES ======================
	\pagenumbering{arabic}	
	
	%====== Objectif : Protection des modules ======%

	%====== Fonctionnalité : Lecture de la tension des modules  ======%
	
	%====== Fonctionnalité : Lecture de du courant  ======%
	
	%====== Fonctionnalité : Lecture de la température  ======%
	
	%====== Fonctionnalité : Détection des seuils et sécurisation de la batterie  ======%
	
	
	%====== Objectif : Balancement des modules ======%
	

	%====== Objectif : Compatibilité avec le BMS présentement utilisé ======%
	
	
	%====== Objectif : Facilité les manipulations lors des vérifications techniques ======%
	
	
\end{document}



