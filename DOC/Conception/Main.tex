\documentclass[12pt,letterpaper]{article}
\usepackage{hyperref}
\hypersetup{colorlinks=true,allcolors=black}
\usepackage[utf8]{inputenc}
\usepackage[frenchb]{babel}
\usepackage[T1]{fontenc}
\usepackage{ae,aecompl}
\usepackage{pslatex}
\usepackage[top=2cm, bottom=2cm, left=2cm, right=2cm]{geometry}
\usepackage{hhline}
\usepackage{amsmath}
\usepackage{pgfplots}
\pgfplotsset{compat=newest}
\usepackage{multicol}
\usepackage{graphicx}
\graphicspath{{Images/}}
\hyphenpenalty=100
\usepackage{caption}
\usepackage{enumitem}
\usepackage{textcomp}
% Pour multicol (Align top)
\raggedcolumns
% Pour enlever l'indentation et avoir un espace entre les paragraphes
%\usepackage[parfill]{parskip}
\usepackage{float}
\floatstyle{plaintop}
\restylefloat{table}


\begin{document}
	
	%====== Page de presentation ======%
	\hypersetup{pageanchor=false}
	
%====== Page de presentation ======%

\title{Eclipse X - BMS}
\author{
	Daigneault-St-Arnaud, Christian, DAIC30099006
}
\newcommand{\cours}{ }
\newcommand{\prof}{ }

\makeatletter
\begin{titlepage}
	\begin{center}
	\pagenumbering{gobble}
	
	{\Huge \@title}\\ 
	\vspace{3cm}
	{\large Par \\
		\vspace{0.5cm}
		\@author \\
		\vspace{3cm}
		%\cours \\
		\vspace{0.5cm}
		%\prof \\
		\vspace{3.5cm}
		\@date \\
		\vspace{3.5cm}
		\'{E}COLE DE TECHNOLOGIE SUP\'{E}RIEURE \\
		UNIVERSIT\'{E} DU QUÉBEC
	}		
	\end{center}

	
\end{titlepage}
\makeatother




	\newpage
	%====== Table des matieres ======%
	\pagenumbering{roman}
	\hypersetup{pageanchor=true}
	\tableofcontents
	%\listoftables
	%\listoffigures	
	%\newpage


%====================== INCLUSION DES PARTIES ======================
	\pagenumbering{arabic}
	
	%====== Objectif : Protection des modules ======%
	\section{Protection des modules}
	\paragraph*{}
	La protection des modules consiste à détecter une surcharge, une décharge excessive, un courant trop élevé et une température trop élevée. Suite à la détection d'une faute, le système de protection doit amener la batterie à un état sécuritaire indépendamment du conducteur. La réglementation exige que des indicateurs de fautes soient installés à l'extérieur du véhicule et sur le tableau de bord. La faute doit être verrouillée et cette dernière doit seulement être effacée manuellement lorsque le véhicule n'est pas en mouvement et que la faute n'est plus présente \cite{cahier_charge}.

	\paragraph*{}
	Pour détecter les différentes fautes, il est nécessaire d'avoir une lecture de la tension de chaque module, une lecture de la température des modules, une lecture du courant de la batterie et un algorithme de détection pour éviter de déclencher la protection pour un faux positif. Pour amener la batterie dans un état sécuritaire et indiquer qu'il y a une faute, le système doit avoir un bus de communication et un contrôle de l'état de la batterie très robuste. La réinitialisation du système de protection requiert une interface accessible et un algorithme qui vérifie que les conditions sont adéquates.




	%====== Fonctionnalité : Lecture de la tension des modules  ======%
	
\subsection{Lecture de tension des modules}
	\paragraph*{}
	Nous désirons avoir une lecture de tension très précise sur toute la plage d'utilisation des modules. Le circuit doit consommer un minimum de courant, être robuste, modulable et précis. Le circuit doit aussi faciliter la vérification technique.
	
	\subsubsection*{Solution commercial}
	\paragraph*{}
	Il existe sur le marché plusieurs circuit qui s'occupe de lire la tension des modules et de communiquer l'information à un microcontrôleur. Les fonctionnalités et le nombre de modules supportés varient d'un vendeur à l'autre. Les points communs sont : 

	\begin{multicols}{2}
		\begin{itemize}
			\item[$\bullet$] Le circuit peut être alimenté par les modules;
			\item[$\bullet$] Consommation de courant de quelque mA lorsque le circuit prend les mesures et quelque $\mu$A lorsqu'il est en veille;
			\item[$\bullet$] Solution compacte;
			\item[$\bullet$] Lecture précise;
			\item[$\bullet$] Les modules doivent être branchés dans l'ordre;
			\item[$\bullet$] Économique;
			\item[$\bullet$] Fonctionnement bien documenté.
		\end{itemize}
	\end{multicols}

	\paragraph*{}
	Cette solution est réalisable et demande surtout de bien lire et comprendre la documentation. Le LTC6804 de Linear Technology a été retenu pour son nombre de modules maximum, son prix et sa simplicité d'implémentation. Cette solution répond à la majorité des spécifications mais elle n'apporte cependant rien de nouveau au système présentement utilisé et elle ne facilite pas la vérification technique. 
	
	\newpage
	
	\subsubsection*{Isolation des lectures}
	\paragraph*{}
	La vérification technique sera beaucoup plus facile si les différentes lectures de tensions sont isolées. Ce point est décrit dans la section Facilité les manipulations lors des vérifications techniques. 
	
	\paragraph*{}
	L'isolation amène plusieurs avantages :
	
	\begin{itemize}
		\item[$\bullet$] Les modules n'ont plus besoin d'être branchés en ordre, ce qui élimine ce risque d'erreur humaine;
		\item[$\bullet$] Il est possible de débrancher une seule cellule pour venir ensuite la remplacer par une alimentation variable;
		\item[$\bullet$] Le filage dans la batterie peut être mieux organisé et optimisé.
	\end{itemize}

	\paragraph*{}
	Cette solution comporte cependant plusieurs désavantage :
	
	\begin{itemize}
		\item[$\bullet$] Consomme plus de courant (quelque mA par module lors des lectures);
		\item[$\bullet$] Demande plus de composantes;
		\item[$\bullet$] Plus difficile à implémenter;
		\item[$\bullet$] Plus dispendieux.
	\end{itemize}
	
	\paragraph*{}
	Ces différents désavantage ne sont pas majeur dans le contexte d'Éclipse, la consommation de courant reste insignifiante comparé à ce que le moteur et le reste des circuits consomment. Le nombre de composant peut être limité en utilisant la bonne topologie et le prix de la carte peut être plus élevé tant qu'il respecte le budget. 
	
	\subsubsection*{Circuit analogique}
	\paragraph*{}
	Une solution relativement simple et qui ne comporte pas beaucoup de composante est montré à la figure \ref{fig:HCNR201}. Ce circuit très compacte et simple n'est pas assez précis puisque le gain entre les deux photodiodes varie de 5\% pour le HCNR201. Cette erreur fait en sorte que le circuit ne répond pas aux spécifications de +/- 10mV.  
	
	\begin{figure}[H]
		\centering
		\includegraphics[scale = 0.5]{Images/Analogique.png}
		\caption{Circuit de lecture de tension isolé analogique \cite{HCNR201}}
		\label{fig:HCNR201}
	\end{figure}

	\subsubsection*{Lecture d'un voltage de référence avec un ADC}
	\paragraph*{}	
	
	
	
	
	
	
	
	
	
	
	%====== Fonctionnalité : Lecture de du courant  ======%
	% Hall effect
%- shunt avec ADC sur le main board
%- shunt avec conversion sur board a part

\subsection{Lecture de courant}
	\paragraph*{}
	La lecture du courant est essentielle pour éviter que la batterie ne se charge ou décharge excessivement. Cette information va aussi être utilisé pour l'estimation de l'état de charge, fonctionnalité qui va être implémenté après le cours ELE-791. Pour que l'estimation de charge soit précise sans accumulé une grande erreur, la lecture de courant ce doit d'être la plus précise sur toute la plage d'utilisation. L'objectif est donc d'avoir une précision qui beaucoup plus grande que le 2\% des spécifications. 
	
	\paragraph*{}
	Eclipse IX utilise présentement la première technique. La résistance est cependant loin de la carte qui prend la mesure, ce qui pourrait causé certains problèmes étant donné que les deux fils ont des signaux avec de très faible voltage qui sont lu par un amplificateur avec une impédance élevé. Afin d'avoir une lecture robuste et mieux protégé du bruit, le circuit de lecture de courant aura sa propre carte au lieu d'être sur la carte maître. Les long fils auront ainsi la communication qui est beaucoup plus robuste au bruits. 
	
	\paragraph*{}
	Deux différentes techniques on été envisagé pour mesurer le courant. La première est de lire la tension au borne d'une résistance en série avec la batterie et la deuxième est d'utiliser un capteur à effet Hall. 
	
	
	\subsubsection*{Lecture de la tension au borne d'une résistance en série avec la batterie}
	\paragraph*{}
	Les résistances "shunt" ont une très grande précision ($\pm 0.1 \% , \pm 0.5 \%$), elle sont facile à interfacer et facile à remplacer lorsque les spécifications changent. Un amplificateur avec une grande impédance d'entrée est nécessaire pour amener le signal analogique de très faible amplitude entre 0V et le voltage de référence de l'ADC. L'amplificateur doit avoir un gain très précis, être thermiquement stable et avoir une bonne atténuation du bruit en mode commun. Le bruit en mode commun peut être réduit en ayant le circuit de lecture référencé à la borne négative de la batterie et en plaçant la résistance en série avec cette même borne.
		
	\paragraph*{}
	Pour avoir une bonne lecture avec un minimum de bruit, une configuration avec l'amplificateur avec une sortie différentiel connecté à un ADC avec une entrée différentiel serait la plus robuste et précise puisque le signal sera au dessus du bruit de fond (noise floor). L'amplificateur PGA281 et le convertisseur analogue à numérique ADC141S626 de Texas Instrument possède les différentes caractéristiques recherchées. La résistance de 250 $\mu \Omega$ d'Éclipse IX serait réutilisée pour Éclipse X.
	
	\subsubsection*{Senseur à effet Hall}
	\paragraph*{}	
	Un capteur à effet Hall à l'avantage de ne pas être intrusif et d'être isolé. Les senseur ratiométrique en boucle ouverte ont une bonne précision qui correspond aux spécifications, il ne sont pas dispendieux et ils sont très facile à interfacer. Le HO 100-S de LEM n'a pas besoin d'amplificateur et peut ainsi être branché directement sur l'entrée du ADC. Les effets Hall on besoin de calibration pour éliminer le décalage lorsqu'il n'y a aucun courant. Ces senseurs ont aussi le désavantage d'avoir certaines non-linéarité dans le gain et un hystérésis. Une bonne caractérisation et calibration du senseur réduirait grandement ces inconvénient.
	
	\subsubsection*{Choix final}
	\paragraph*{}
		
	%====== Fonctionnalité : Controle Contacteur  ======%
	
\subsection{Contrôle des contacteurs}

	\paragraph*{}
	Pour amener la batterie dans un état sécuritaire, il faut pouvoir la débrancher de l'alimentation. Le système utilise donc des contacteurs à haute puissance pour contrôler la connection de la batterie avec le reste des charges. Le module maître se charge de contrôler ces contacteurs. Il s'occupe d'enclencher les contacteurs lors de la mise en marche du système lorsqu'aucune faute n'est détectée. De plus, lorsque le module maître reçoit un message d'erreur, il doit s'assurer de déconnecter la batterie pour protéger les modules en cas de fautes.

	\paragraph*{}
	Le système est composé de trois contacteurs, soient deux contacteurs principaux et un pour les MPPTs. L'un des contacteurs principaux est connecter à la borne positive de la batterie et l'autre à la borne négative. On peut voir à la figure suivante un schéma générale de la disposition des contacteurs. 
	
	\begin{figure}[H]
		\centering
		\fbox{\includegraphics[width=0.6\linewidth]{Images/LithiumBalanceContactorDiagram}}
		\caption[Diagramme général des contacteurs]{Image tirée du guide utilisateur du BMS de LithiumBalance \cite{Lithium_Balance}}
		\label{fig:lithiumbalancecontactordiagram}
	\end{figure}

	\paragraph*{}
	L'enclenchement des contacteurs se fait avec une tension de 12V, qui doit être appliquée à ses bornes. Puisque c'est le micro-contrôleur qui envoi la commande d'ouverture ou de fermeture, il doit y avoir une interface de protection entre le 12V aux bornes du contacteur et le 3.3V de la patte GPIO. De façon générale, il n'est pas conseillé d'utiliser des actionneurs mécanique, tel qu'un relais, pour activer les contacteurs puisqu'ils sont souvent victimes de défaille. 	 

	\subsubsection*{Choix des contacteurs:}
		\paragraph*{}
		Une commandite d'Autobus Lion a permis au club Éclipse de se procurer trois contacteurs de marque GIGAVAC\_HX21. Il faut donc que le module maître puisse interfacer ces contacteurs. Le système de batterie d'Éclipse 9 utilise des contacteurs EV200 et ceux-ci seront utilisés comme pièce de remplacement. La conception du circuit de contrôle doit être compatible avec les deux modèles.

		\paragraph*{}
		Les spécifications des contacteurs Gigavac montrent que le courant consommé lors de l'ouverture de l'inductance est de 4.3 A. En ajoutant une marge de protection d'un facteur de 1.2, les composantes doivent soutenir un courant de jusqu'à 5.16 A pendant 75 ms. 

		\paragraph*{}
		De plus, ce modèle de contacteur possède un deuxième relais interne qui peut être utilisé pour l'interverrouillages (interlock), beaucoup utilisé dans le domaine des véhicules. Dans le cas du présent système, ce relais sera utilisé pour activer un témoin lumineux afin d'avertir l'utilisateur de l'état des contacteurs. Avec les contacteurs EV-200, il faudra court-circuiter deux positions sur le connecteur allant sur le module maître pour activer les témoins lumineux.

		\paragraph*{}
		Également, le circuit de contrôle doit répondre aux critères suivants afin d'être jugés sécuritaires. (source: A systems approach to Lithium-Ion Battery Management )


		\begin{itemize}
			\item Prévenir les défaillances qui causent les contacteurs à ouvrir alors qu'ils devraient être fermés
			\item Prévenir les défaillances qui causent les contacteurs à fermer alors qu'ils devraient être ouverts
			\item Prévenir les défaillances qui causent une connection de la batteries à travers une charge	capacitive	
		\end{itemize}


	\subsubsection*{Choix du circuit de contrôle:}
		\paragraph*{}
		Plusieurs solutions ont été évalué pour contrôler les contacteurs. Parce que cette section est très critique et qu'il ne faut absolument pas que durant la compétition, une défaillance met en panne la voiture solaire. C'est pourquoi le budget n'est pas un critère de sélection mais plutôt une solution qui est le plus sécuritaire possible.

		Schéma général d'un contacteur 

		\paragraph*{Solution 1: Contrôle par optocoupleurs}
			
		Puisque les activateurs ne doit pas être de nature mécanique. On peut utiliser des optocoupleurs (Solid State Relays) puisqu'ils sont isolés mécaniquement. Le microcontrôleur doit seulement alimenter une DEL de l'optocoupleur pour l'enclencher. Aussi, dans l'éventualité que le micro-contrôleur ait une défaillance, l'optocoupleur s'ouvrira, empêchant ainsi le contacteur de rester fermé. Il faut par contre. choisir un optocoupleur qui puissent accepter jusqu'à 5.16 A, ce qui limite beaucoup le choix de composantes.
		
		\begin{table}[H]
			\centering
			\caption{Comparaison des optocoupleurs}
			\label{my-label}
			\begin{tabular}{|p{3cm}|p{3cm}|p{3cm}|p{3cm}|}
				\hline
				\textbf{Optocoupleurs} & \textbf{Courant de charge} & \textbf{Résistance} & \textbf{Prix}
				\\ \hhline{|=|=|=|=|}
				CPC1907B &
				6 A &
				60 mOhm &
				9.01 \$
				\\ \hline
				CPC1709J &
				10 A &
				50 mOhm &
				10 \$
				\\ \hline
				CPC1918J &
				5.25 A &
				100 mOhm &
				14.83 \$ 		
				\\ \hline
			\end{tabular}
		\end{table}
		
		Le prix de ces composantes est similaire et leur courant de charge est assez robuste. Le choix s'est donc faite sur les CPC17907B en raison de leur profil plat qui était plus avantageux avec le placement des pièces sur le module maître.
		
		\begin{figure}[H]
			\centering
			\fbox{\includegraphics[width=0.4\linewidth]{../Images/ContactorSol1}}
			\caption[Solution 1]{Schéma solution 1}
			\label{fig:contactorsol1}
		\end{figure}
		
		
		\paragraph*{Solution 2: Contrôle par Mosfet en série}
		Une façon d'assurer la protection d'un circuit est de rajouter de la redondance. Plusieurs Mosfet sont misent en série pour augmenter les chances d'ouverture de circuit en cas de défaillances. Plus il y a de composantes en série, meilleures sont les chances mais on augmente aussi le prix en composantes. Chaque Mosfet doit être contrôlé individuellement par une patte du micro-contrôleur. Aussi, il est important de s'assurer que la résistance interne des Mosfets soit constante d'une composante à l'autre. C'est pour éviter l'échauffement du Mosfet ayant la résistance interne la plus élevée, pouvant causer des défaillances.
		
		\begin{figure}[H]
			\centering
			\fbox{\includegraphics[width=0.4\linewidth]{../Images/ContactorSol2}}
			\caption[Solution 2]{Schéma solution 2}
			\label{fig:contactorsol2}
		\end{figure}
		
		

		\paragraph*{Solution 3: Circuit d'alimentation à double actionneurs}
		Cette solution provient du livre A systems approach to Lithium-Ion Battery Management\cite{System_Approach}. Le livre propose d'utiliser circuit d'alimentation à double actionneurs, soit un actionneurs sur le pôle négatif et un sur le pôle positif. Un Mosfet de type-P alimente le côté positif (High-side) et un mosfet de type-N est ferme le circuit du côté négatif (low-side) ce qui offre une sécurité supplémentaire si une des deux composantes venait à faillir. Puisqu'il y a trois contacteurs dans le système, cette solution requiert d'utiliser un seul Mosfet type-P pour alimenter le côté positif des contacteurs et trois Mosfet type-N pour le côté négatif de chaque relais.

		\begin{figure}[H]
			\centering
			\fbox{\includegraphics[width=0.4\linewidth]{../Images/ContactorSol3}}
			\caption[Solution 3]{Schéma solution 3}
			\label{fig:contactorsol3}
		\end{figure}

		
		\paragraph*{Choix final}
		La solution retenue est un mélange de la solution 1 et de la solution 3. En ayant un circuit d'alimentation à double actionneurs, on évite d'avoir de la redondance. On remplace ainsi les mosfets de cette solution par les optpocoupleurs de la solution 1. De cette façon, on s'assure de la plus haute protection tout en ayant le minimum de composantes. Puis, puisqu'on veut éviter de tirer trop de courant du micro-contrôleur, on utilise un transistor BJT s'activant avec un faible courant. Le BJT peut ensuite alimenter l'optocoupleur avec l'alimentation 3.3V directement.

		\begin{figure}[H]
			\centering
			\fbox{\includegraphics[width=0.4\linewidth]{../Images/ContactorSolFinal}}
			\caption[Choix final]{Choix final}
			\label{fig:contactorsolfinal}
		\end{figure}
		


	\subsubsection*{Circuit de précharge:}
		\paragraph*{}
		Le contacteur principal sur le pôle positif de la batterie, ainsi que le contacteur du MPPT, possèdent un circuit de précharge. La précharge permet à la batterie de se connecter à une large charge capacitive. Si la batterie est connectée directement avec une charge capacitive déchargée, le courant transitoire ne sera limitée que par l'impédance interne de la battery, de la charge et des contacteurs, ce qui ne sera généralement pas assez pour prévenir un courant potentiellement dangereux. Un contacteur qui s'enclenche lorsque la différence de potentiel est trop élevée pourrait le soude et empêcher son ouverture.
		
		\paragraph*{}
		Pour prévenir ce problème, on ajoute une résistance en série avec un relais, installés en parallèle du contacteurs. Avant d'enclencher le contacteur, un courant limité passe par la résistance, et permet à la tension de monter de façon exponentielle. On compare ensuite la lecture de la charge avec la tension des batteries. Lorsque la tension sur la charge est assez élevée et que la différence de potentiel aux bornes du contacteurs est négligeable, le contacteur peut être fermé de façon sécuritaire.
		

		
		\paragraph*{}		
		La résistance de précharge est dimensionné pour dissiper la puissance circulant dans la résistance. La tension maximale de la batterie est de 160 V, et avec une résistance de 100 Ohm, on doit dissiper une puissance de 256 W avec un courant de 1.6 A. Une résistance dissipant autant de puissance risque d'être trop grosse. Pour réduire la puissance il faut mettre une résistance de 270 Ohm en série pour avoir un courant de une puissance de 94.81 W. Équipée d'un dissipateur de chaleur, une résistance de 100W doit être choisie.
		
		\begin{figure}[H]
			\centering
			\includegraphics[width=0.4\linewidth]{../Images/CalculPuissancePrecharge}
		\end{figure}
		
		
		Tableau de la sélection de la résistance
		
		AP101 270R J 
		Ohmite
		100 W
		12.47 \$
		
		C247-025-1AE
		Ohmite
		TO-247
		6.20 \$

		\paragraph*{}		
		Le temps de précharge doit être judicieusement choisi s'assurer que la charge capacitive est assez chargée. Pour ce faire, il faut calculer Tau=Résistance de précharge x la capacitance du circuit à haute tension. Puisqu'un condensateur est considéré chargé à 5 x tau, on trouve le temps total et on ajoute 0.5 seconde comme marge de sécurité. On obtient ainsi le temps minimum de précharge. En se fiant à ce calcul, une durée de 0.65 seconde sera nécessaire si l'on estime la charge capacitive à 3 uF.

		Image précharge
		
		\paragraph*{}		
		Le relais doit supporter un courant de 0.6 A. Sa bobine devrait également consommer le moins possible. Un relais d'utilisation général à été choisi.
		
		
		ALQ312
		10A
		SPST-NO
		16.7 mA
		3.34 \$
		
	
		\paragraph*{}	
		Pour éviter de polariser à l'inverse les contacteurs avec la précharge, une diode est ajoutée au circuit. Elle doit bloquer une tension minimum de 160 V et pouvoir supporter un courant d'au moins 0.6 A. 
		
		\paragraph*{}			
		Ensuite, pour éviter qu'un courant trop élevé ne vienne endommager les composantes, on ajoute une fusible de 600mA à action retardée. 
		
		
		\begin{figure}[H]
			\centering
			\fbox{\includegraphics[width=0.7\linewidth]{Images/CircuitPrecharge}}
			\caption[Schéma circuit résistance de précharge]{}
			\label{fig:circuitprecharge}
		\end{figure}
			
		
	%====== Fonctionnalité : Lecture de la température  ======%
	
	%====== Fonctionnalité : Détection des seuils et sécurisation de la batterie  ======%

	%====== Fonctionnalité : Avertir l'utilisateur  ======%
	
\subsection{Avertie l'utilisateur}

	\paragraph*{}
	Message CAN vers board des lumières pour alumière externe
	LED driver erreur externe pour cockpit
	Led d'erreur sur le master board
		
	%====== Fonctionnalité : Réinitialisation  ======%
	\input{Fonctionnalites/ReinitialisationSystème.tex}
	
	%====== Objectif : Balancement des modules ======%
	

	%====== Objectif : Compatibilité avec le BMS présentement utilisé ======%
	
	
	%====== Objectif : Facilité les manipulations lors des vérifications techniques ======%

	%====== Architecture ======%
	\section{Architecture}
	
	\subsection{Architecture générale}
		\begin{figure}[H]
			\centering
			\fbox{\includegraphics[width=0.7\linewidth]{"Images/Architecture_generale"}}
			\caption{Architecture générale du système de protection et de contrôle}
			\label{fig:architecture_generale}
		\end{figure}
	
	\subsection{Architecture du module de lecture de courant}
		\begin{figure}[H]
			\centering
			\fbox{\includegraphics[width=0.7\linewidth]{Images/Architecture_LectureCourant}}
			\caption{Architecture du module de lecture de courant}
			\label{fig:architecturelecturecourant}
		\end{figure}
	\subsection{Architecture du module maître}
	
		\begin{figure}[H]
			\centering
			\fbox{\includegraphics[width=0.7\linewidth]{"Images/Architecture_Master"}}
			\caption{Architecture du module maître}
			\label{fig:architecture_master}
		\end{figure}

	\subsection{Architecture du module esclave}
	
	\begin{figure}[H]
		\centering
		\fbox{\includegraphics[width=0.7\linewidth]{"Images/Architecture_Slave"}}
		\caption{Architecture du module esclave}
		\label{fig:architecture_slave}
	\end{figure}


	
	\subsection{Choix technologiques}
	\subsubsection*{Système d'exploitation en temps réel}
		\paragraph*{}
		Le système d’exploitation en temps réel (RTOS) est utilisé pour rendre le système plus robuste. Il permet de facilement prioriser les tâches les plus importantes du système ainsi que d’améliorer son déterminisme. De plus, il rend la modification et la maintenance du code très simple grâce à la segmentation de l’application en tâches distinctes. Plusieurs RTOS sont disponibles et offrent des caractéristiques similaires. Les grands différenciateurs sont le coût, la robustesse (évaluée selon son utilisation dans des systèmes critiques), la licence utilisée pour le code et la facilité d’utilisation.

		\begin{table}[H]
			\centering
			\caption{Comparaison des RTOS}
			\label{my-label}
			\begin{tabular}{|p{3cm}|p{3cm}|p{3cm}|p{3cm}|p{3cm}|}
				\hline &
				\textbf{Prix} & \textbf{Robustesse} & \textbf{Licence} & \textbf{Facilité}
				\\ \hhline{|=|=|=|=|=|}
				uC/OS-III &
				Gratuit (pour les étudiants et hobbyistes) &
				Certifié aérospatial, médical et industriel.
				Utilisé dans le « rover » Curiosity de la NASA. &
				Propriétaire mais à source ouverte. &
				Plateforme de développement déjà construite et prêt à être utilisée.
				Connu des membres de l’équipe.
				Documentation exceptionnelle.
				\\ \hline
				FreeRTOS &
				Gratuit &
				Aucune certification. Pour avoir une version certifiée, il faut utiliser une version payante (SafeRTOS). &
				FreeRTOS Open Source License (GNU GPL modifiée). &
				Projet de base facilement générable à l’aide d’outils.
				Bonne documentation.
				\\ \hline
				RTEMS &
				Gratuit &
				Aucune information sur les certifications disponible.
				Utilisé dans l’espace par la NASA et l’ESA. &
				RTEMS License (GNU GPL modifiée). &
				Complexe.
				Bonne documentation.                                                                              \\ \hline
			\end{tabular}
		\end{table}

		\paragraph*{}
		Notre choix s’est arrêté sur uC/OS-III, surtout à cause de la facilité d’utilisation étant donné qu’une plateforme était déjà disponible et utilisée dans le club Éclipse. Le fait de réutiliser cette même plateforme nous fait gagner du temps et permettra à d’autres membres du club de pouvoir facilement intégrer le projet et l’améliorer au fil du temps. La robustesse du RTOS était aussi un point majeur, puisque le BMS est une application critique pour la vie.

	\subsubsection*{Protocole de communication}
		\paragraph*{}
		Nous avions le choix d’interfacer les périphériques externes avec plusieurs protocoles de communication. Les deux plus populaires sont le I2C et le SPI. Le I2C nécessite seulement deux traces pour relier tous les périphériques externes avec le microcontrôleur aux dépens d’une plus grande complexité dans la façon de communiquer. Quant au SPI, c’est le contraire. La façon de l’utiliser est très simple, mais il faut trois traces plus une supplémentaire pour chaque périphérique externe. De plus, le SPI peut aller a des vitesses plus grandes que le I2C.

		\paragraph*{}
		Notre choix s’est arrêté sur le SPI à cause de sa simplicité d’utilisation. Plus le code pour le gérer est simple, plus il est facile à tester et donc plus il a de chance d’être parfaitement robuste rapidement. Le microcontrôleur choisi peut accommoder le nombre de traces supplémentaires nécessaires, ce qui n’entraîne donc aucun coût supplémentaire relié au désavantage du SPI par rapport au I2C.

	\subsubsection*{Interruption et interrogation (polling)}
		\paragraph*{}
		Nous avons décidé de minimiser l’utilisation des interruptions à ce qui est vraiment nécessaire étant donné que lorsqu’une interruption est déclenchée, le système priorise le traitement de celle-ci par rapport au reste des fonctions. Les interruptions sont utiles pour obtenir un temps de réponse très rapide pour les fonctions prioritaires et de la précision pour les bases de temps (time base). Par contre, dépendamment des situations, elles peuvent ajouter une charge non négligeable sur le système et pourraient faire en sorte de le bloquer complètement s’il y avait une erreur de programmation dans les fonctions qui les prennent en charge. Si l’interruption est externe, comme pour un bouton, il est aussi dangereux qu’elle soit déclenchée sans le vouloir à cause d’un « glitch ».

	\subsubsection*{Microcontrôleur}
		\paragraph*{}
		Une multitude de marques de microcontrôleur existe sur le marché. Une des plus populaires est STMicroeletronics avec sa gamme STM32. Celle-ci est basée pour l’architecture la plus répandue pour le développement embarqué, soit l’ARM Cortex-M.

		\paragraph*{}
		Le choix de cette gamme repose sur plusieurs facteurs distinctifs. Premièrement, des cartes d’évaluation sont offertes à très faible coût, ce qui permet de commencer le développement avant d’avoir reçu le vrai matériel. Ensuite, une suite de logiciel permettant d’utiliser les périphériques internes très facilement est disponible. De plus, une plateforme utilisant cette gamme de microcontrôleurs est déjà utilisée par le club scientifique Éclipse. Il y a donc déjà des connaissances acquises quant à son utilisation.

	\subsection{Logiciel}
	\paragraph*{}
	L'architecture logicielle des trois modules est présentée ici sous forme de pile montrant ainsi les interdépendances entre chaque couche. La première couche est pour l'applicatif. On y trouve les différentes tâches que le module doit effectuer. Ensuite viennent la couche des "middlewares", puis celle des services. Les services sont des librairies logicielles contextualisées qui utilisent les pilotes pour effectuer des tâches de plus haut niveau. Les trois dernières couches sont celles des pilotes, de l'abstraction matérielle et des périphériques matériels internes.
	\subsubsection*{Module maître}
		\noindent
		Interruptions : Base de temps du HAL, base de temps du RTOS, CAN \#1 (Tx et Rx),  CAN \#2 (Tx et Rx) \\
		Interrogations : ADC interne thermistor (x1), Bouton de reprise d’erreur
		\begin{figure}[H]
			\centering
			\includegraphics[scale=0.5]{Images/Logiciel_Master.png}
			\caption{Logiciel du module maître}
			\label{fig:logiciel_master}
		\end{figure}
	\subsubsection*{Module esclave}
		\noindent
		Interruptions : Base de temps du HAL, base de temps du RTOS, CAN \#1 (Tx et Rx) \\
		Interrogations : ADC externe cellules (x8), ADC interne thermistor (x2)
		\begin{figure}[H]
			\centering
			\includegraphics[scale=0.5]{Images/Logiciel_Slave.png}
			\caption{Logiciel des modules esclaves}
			\label{fig:logiciel_slave}
		\end{figure}
	\subsubsection*{Module de lecture de courant}
		\noindent
		Interruptions : Base de temps du HAL, base de temps du RTOS, CAN \#1 (Tx et Rx) \\
		Interrogations : ADC externe shunt
		\begin{figure}[H]
			\centering
			\includegraphics[scale=0.5]{Images/Logiciel_Current_Sense.png}
			\caption{Logiciel du module de lecture de courant}
			\label{fig:logiciel_current_sense}
		\end{figure}

	\subsection{Diagrammes fonctionnels}
	\paragraph*{}
	\subsubsection{Module maître}
		\paragraph*{}
	\subsubsection{Module esclave}
		\paragraph*{}
	\subsubsection{Module de lecture de courant}
		\paragraph*{}

%====================== Bibliographie ======================

	\begin{thebibliography}{10}
	
	\bibitem{cahier_charge} 
	Christian Daigneault-St-Arnaud, Julien Gagnon-Bourassa, Olivier Cusson-Larocque. \textit{Cahier des charges, Système de protection et de gestion de batterie Li-ion}. \par	
	
	\bibitem{HCNR201} 
	Avago Technologies. \textit{HCNR200 and HCNR201 Data Sheet}.\\ \texttt{\url{https://docs.broadcom.com/docs/AV02-0886EN}}\par

	
	\end{thebibliography}	
	
\end{document}



