\section{Protection des modules}
	\paragraph*{}
	La protection des modules consiste à détecter une surcharge, une décharge excessive, un courant trop élevé et une température trop élevée. Suite à la détection d'une faute, le système de protection doit amener la batterie à un état sécuritaire indépendamment du conducteur. La réglementation exige que des indicateurs de fautes soient installés à l'extérieur du véhicule et sur le tableau de bord. La faute doit être verrouillée et cette dernière doit seulement être effacée manuellement lorsque le véhicule n'est pas en mouvement et que la faute n'est plus présente \cite{cahier_charge}.
	
	\paragraph*{}
	Pour détecter les différentes fautes, il est nécessaire d'avoir une lecture de la tension de chaque module, une lecture de la température des modules, une lecture du courant de la batterie et un algorithme de détection pour éviter de déclencher la protection pour un faux positif. Pour amener la batterie dans un état sécuritaire et indiquer qu'il y a une faute, le système doit avoir un bus de communication et un contrôle de l'état de la batterie très robuste. La réinitialisation du système de protection requiert une interface accessible et un algorithme qui vérifie que les conditions sont adéquates. 
	
	
	