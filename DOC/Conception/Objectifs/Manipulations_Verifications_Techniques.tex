\section{Manipulations et vérifications techniques}
\paragraph*{}
L'installation du système de protection dans la batterie doit être sécuritaire autant pour l'utilisateur, pour la batterie que pour les circuits. Les différents modules doivent être protégés pour tolérer des erreurs d'inattention et de mauvaise manipulation. La vérification technique lors de la compétition doit prendre moins de 12 manipulations et ses manipulations ne doivent pas être compliquées à exécuter.

	\subsection{Protections}
	\paragraph*{}
	Le module qui présente des risques lors des manipulations est le module esclave. Le système de protection présentement utilisé exige que des fusibles soient installés en série avec chaque entrée des mesures de tension. Les fils sont très longs et désordonnés et l'inversion de deux connexions peut avoir un effet catastrophique. En utilisant des modules de lecture de tension isolés, un circuit de protection contre une polarité inverse et un fusible directement au connecteur, il est possible d'avoir un filage mieux organisé et beaucoup plus sécuritaire.

	\subsection{Connecteurs}
	\paragraph*{}
	Les circuits doivent être manipulables facilement et rapidement puisqu'en compétition les membres de l'équipe sont souvent pressés et fatigués. Les connecteurs doivent supporter des gestes brusques, doivent s'insérer rapidement et ne doivent pas permettre d'être branchés à l'envers. Les connecteurs à angle droit avec une broche pour le support mécanique et un détrompeur de la série 6490 de Wurth répondent aux critères.

	\subsection{Vérifications techniques}
	\paragraph*{}
	Afin de faciliter les vérifications techniques, les lectures de tension sont isolées pour seulement avoir à remplacer un module par une alimentation variable en enlevant seulement un connecteur. La validation de la lecture de courant demande de débrancher la résistance "shunt" puisqu'elle est trop petite et qu'il est impossible pour l'alimentation de fournir assez de courant. Un petit circuit avec un commutateur DPDT pour inverser la polarité et un diviseur de tension remplacera la résistance de lecture. La vérification des senseurs de température consiste seulement à sortir la thermistance et à la chauffer avec un séchoir à cheveux.