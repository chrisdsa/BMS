\subsection{Ré-initialisation du système}

	\paragraph*{}
	Lorsqu'une faute est détectée par le système de protection de batterie, les contacteurs restent ouverts. Pour ré-initialiser le BMS, il faut que l'utilisateur effectue cette opération de façon manuelle. La ré-initialisation peut se faire de trois manières différentes.
		
	\subsubsection{Communication CAN}
		\paragraph*{}
		Premièrement, on peut réinitialiser le système en envoyant un message CAN au module maître par un autre circuit ou avec un ordinateur branché sur le réseaux CAN. 

	\subsubsection{Communication RS232}
		\paragraph*{}	
		Deuxièmement, on peut utiliser la communication RS232 en branchant un ordinateur personnel sur le port de communication du module maître. On peut ainsi envoyer une commande de ré-initialisation avec l'aide d'un terminal tel que Putty.
				
	\subsubsection{Interrupteur sur le module maître}
		Troisièmement, une touche contact est présente sur le module maître pour manuellement réinitialiser le BMS. Cette touche est aussi utilisée comme interrupteur tout usage lors du dépannage de la carte.