% Hall effect
%- shunt avec ADC sur le main board
%- shunt avec conversion sur board a part

\subsection{Lecture de courant}
	\paragraph*{}
	La lecture du courant est essentielle pour éviter que la batterie ne se charge ou décharge excessivement. Cette information va aussi être utilisé pour l'estimation de l'état de charge, fonctionnalité qui va être implémenté après le cours ELE-791. Pour que l'estimation de charge soit précise sans accumulé une grande erreur, la lecture de courant ce doit d'être la plus précise sur toute la plage d'utilisation. L'objectif est donc d'avoir une précision qui beaucoup plus grande que le 2\% des spécifications. 
	
	\paragraph*{}
	Eclipse IX utilise présentement la première technique. La résistance est cependant loin de la carte qui prend la mesure, ce qui pourrait causé certains problèmes étant donné que les deux fils ont des signaux avec de très faible voltage qui sont lu par un amplificateur avec une impédance élevé. Afin d'avoir une lecture robuste et mieux protégé du bruit, le circuit de lecture de courant aura sa propre carte au lieu d'être sur la carte maître. Les long fils auront ainsi la communication qui est beaucoup plus robuste au bruits. 
	
	\paragraph*{}
	Deux différentes techniques on été envisagé pour mesurer le courant. La première est de lire la tension au borne d'une résistance en série avec la batterie et la deuxième est d'utiliser un capteur à effet Hall. 
	
	
	\subsubsection*{Lecture de la tension au borne d'une résistance en série avec la batterie}
	\paragraph*{}
	Les résistances "shunt" ont une très grande précision ($\pm 0.1 \% , \pm 0.5 \%$), elle sont facile à interfacer et facile à remplacer lorsque les spécifications changent. Un amplificateur avec une grande impédance d'entrée est nécessaire pour amener le signal analogique de très faible amplitude entre 0V et le voltage de référence de l'ADC. L'amplificateur doit avoir un gain très précis, être thermiquement stable et avoir une bonne atténuation du bruit en mode commun. Le bruit en mode commun peut être réduit en ayant le circuit de lecture référencé à la borne négative de la batterie et en plaçant la résistance en série avec cette même borne.
		
	\paragraph*{}
	Pour avoir une bonne lecture avec un minimum de bruit, une configuration avec l'amplificateur avec une sortie différentiel connecté à un ADC avec une entrée différentiel serait la plus robuste et précise puisque le signal sera au dessus du bruit de fond (noise floor). L'amplificateur PGA281 et le convertisseur analogue à numérique ADC141S626 de Texas Instrument possède les différentes caractéristiques recherchées. La résistance de 250 $\mu \Omega$ d'Éclipse IX serait réutilisée pour Éclipse X.
	
	\subsubsection*{Senseur à effet Hall}
	\paragraph*{}	
	Un capteur à effet Hall à l'avantage de ne pas être intrusif et d'être isolé. Les senseur ratiométrique en boucle ouverte ont une bonne précision qui correspond aux spécifications, il ne sont pas dispendieux et ils sont très facile à interfacer. Le HO 100-S de LEM n'a pas besoin d'amplificateur et peut ainsi être branché directement sur l'entrée du ADC. Les effets Hall on besoin de calibration pour éliminer le décalage lorsqu'il n'y a aucun courant. Ces senseurs ont aussi le désavantage d'avoir certaines non-linéarité dans le gain et un hystérésis. Une bonne caractérisation et calibration du senseur réduirait grandement ces inconvénient.
	
	\subsubsection*{Choix final}
	\paragraph*{}
		