\section{Introduction}

	\subsection{Mise en contexte}
		\paragraph*{Danger des batteries Li-ion:}
		Dans le domaine des énergies renouvelables, de l'électronique mobile et de l'électrification des transports, la batterie au Lithium-Ion est de plus en plus populaire. Elle possède effectivement une des meilleures densités énergétiques comparativement aux autres chimies de batteries. Par contre, elles sont très réactives et une mauvaise utilisation des ces batteries peut créer des dommages considérables. C'est pourquoi ce type de batteries doit posséder un système de protection pour assurer qu'elles fonctionnent à l'intérieur des spécifications prévues par les manufacturiers.
		
		\paragraph*{Système de protection et de gestion de batteries:}
		La protection des batteries se fait d'abord par la lecture de la tension des modules, de leur température et du courant. Ensuite, le système vérifie si toutes ces valeurs respectent les limites. Puis, ces informations sont envoyées à d'autres systèmes sur un réseau de communications. De plus, le système contrôle des contacteurs qui relient la batterie à sa charge et son chargeur.
		
		\paragraph*{Projet Spécial:}		
		Dans le cadre du cours ELE791, les auteurs réalisent un projet destiné à un club étudiant participant aux diverses compétitions d'ingénierie. Or, les membres de ce projet font partie du club étudiant Éclipse, club fabriquant une voiture solaire participant à des compétitions contre d'autres universités à travers le monde. Le but du projet est de faire la conception et la réalisation d'un système de protection et de gestion de batteries de la dixième itération du véhicule solaire. 
		
	\subsection{Rappel des objectifs:}		
		Puisque l'ampleur du projet est colossale, l'équipe s'est fixé des objectifs dans le cadre du projet spécial. Ces objectifs, jugés essentiels, devront être réalisés dans les quatre mois dédiés à ce projet. D'autres objectifs ont déjà été établis, mais ceux-ci seront réalisé une fois le projet spécial terminé, possiblement par d'autres membres d'Éclipse.
		
		Objectifs établis dans le cadre du projet spécial:
		
			\begin{itemize}
				\item[$\bullet$] Protection des modules;
				\item[$\bullet$] Balancement des modules;
				\item[$\bullet$] Compatibilité avec le BMS présentement utilisé;
				\item[$\bullet$] Facilité les manipulations lors des vérifications techniques
			\end{itemize}
				
		Objectifs à réaliser à la suite du projet spécial:
		
			\begin{itemize}
				\item[$\bullet$] Contrôle du système de refroidissement;
				\item[$\bullet$] Calcul de l'état de charge;
				\item[$\bullet$] Limites dynamiques;
				\item[$\bullet$] Surveiller la batterie et envoyer les informations sur un serveur
			\end{itemize}		
	
	\subsection{Description des parties:}
		\paragraph*{}			
		Le document présenté est un cahier de conception qui détaille les différents choix technologiques reliés aux fonctionnalités d'un système de protection de batteries. Il détaille également les circuits électroniques utilisés pour accomplir les objectifs fixés par l'équipe.
		
		\paragraph*{}
		Tout d'abord, l'objectif de la protection des modules est abordé, en expliquant les différentes fonctionnalités pour accomplir cette tâche. Ces fonctionnalités consistent à lire la tension des modules, la lecture de la tension et du courant de la batterie, la température des modules ainsi que la détection des seuils de sécurisation. De plus, la fonctionnalité du contrôle des contacteurs ainsi que leur précharge sont détaillés. Puis, le document fait mention de la façon dont le système signal les erreurs et comment l'utilisateur peut réinitialiser le système.	Ensuite, il est question du balancement des cellules et les choix du concept. Puis, le rapport explique comment le nouveau système est compatible avec l'ancien. De plus, il mentionne comment les manipulations sont facilités lors des vérifications techniques. Finalement, l'architecture des différents modules est détaillée ainsi que les choix technologiques électroniques et logiciels, suivi des diagrammes fonctionnels du système.
	
			
