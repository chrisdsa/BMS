\documentclass[12pt,letterpaper]{article}
\usepackage{hyperref}
\hypersetup{colorlinks=true,allcolors=black}
\usepackage[utf8]{inputenc}
\usepackage[frenchb]{babel}
\usepackage[margin=0.75in]{geometry}
\usepackage{hhline}
\usepackage{amsmath}
\usepackage{pgfplots}
\pgfplotsset{compat=newest}
\usepackage{multicol}
\usepackage{graphicx}
\graphicspath{}
\hyphenpenalty=100
\usepackage{caption}

\usepackage{float}
\floatstyle{plaintop}
\restylefloat{table}

\begin{document}
	
	%====== Page de presentation ======%
	\hypersetup{pageanchor=false}
	%================ Page titre ================

\title{
	\textbf{Plan de test} \\
	\vspace{2cm}
	Système de protection et de gestion de batterie Li-ion	
}
\author{
	Daigneault-St-Arnaud, Christian, DAIC30099006 \\
	Gagnon-Bourassa, Julien, GAGJ23108601 \\
	Cusson-Larocque, Olivier, CUSO09048905	
}
\newcommand{\cours}{ELE791 - Projets spéciaux }
\newcommand{\prof}{Deslandes, Dominic}



\makeatletter
\begin{titlepage}


	\pagenumbering{gobble}
	\centering
	{\Huge \@title}\\ 
	\vspace{3cm}
	{\large Par \\
		\vspace{0.5cm}
		\@author \\
		\vspace{3cm}
		\cours \\
		\vspace{0.5cm}
		\prof \\
		\vspace{3.5cm}
		\@date \\
		\vspace{3.5cm}
		\'{E}COLE DE TECHNOLOGIE SUP\'{E}RIEURE \\
		UNIVERSIT\'{E} DU QUÉBEC
	}
\end{titlepage}
\makeatother





	\newpage
	%====== Table des matieres ======%
	\pagenumbering{roman}
	\hypersetup{pageanchor=true}
	%\tableofcontents
	%\listoftables
	%\listoffigures	
	\newpage
	
	
	%====================== INCLUSION DES PARTIES ======================
	\pagenumbering{arabic}
	
	\section{Contexte et définition du problème}


	\subsection{Danger des batteries Li-Ion}
		L'avènement des énergies renouvelables et de l'électrification des transports amènent leur lot de défis. Notamment, celui d'emmagasinner l'énergie et d'être capable de délivrer des puissances de plus en plus élevées. Une des technologies prévilégiées à cet effet est l'utilisation d'accumulateurs au lithium-ion. Ayant une densité énergétique intéressante, beaucoup de développement technologique est effectué dans ce domaine. Par contre, cette caractéristique recherchée est un coûteau à deux tranchants. En effet, ce type d'accumulateurs nécessite des équipements de protections, sans quoi ils peuvent devenir instables et causer des dommages importants. Les accumulateurs doivent également opérer à l'intérieur des caractérisiques données par le manufacturier, ce qui n'est pas toujours évident à respecter lorsque plusieurs facteurs influencent ces caractérisques. De plus, dans plusieurs applications, l'utilisation de plusieurs centaines de ces accumulateurs sont reliés ensemble pour former une batterie, ajoutant des risques supplémentaires. C'est pourquoi un système de surveillance est nécessaire au bon fonctionnement de la batterie et pour éviter des événements catastrophiques.

		
	\subsection{Système de protection et de gestion de batterie}
		La protection et la gestion de batteries se fait par un système électronique. Il est important de comprendre les différentes composantes formant le système de protection et de gestion de batterie.
	
		\paragraph{}
		\textbf{Cellule:} C'est la plus petite unité composant un système. Les cellules sont des accumulateurs chimiques pouvant avoir une formes cylindriques, prismatiques ou en sachet.  
		
		\paragraph{}
		\textbf{Module:} Les cellules sont regrouper en parralèles pour former des modules. De cette façon, on peut additionner le courant vennant des cellules. Les modules peuvent être remplacer dans un système.
		
		\paragraph{}
		\textbf{Batterie:} On dispose les modules en série pour former la batterie. On augmente ainsi la tension pour l'application voulue.
					
		\paragraph{}
		\textbf{Système de protection:} Cette composante effectue des mesures sur les modules. On mesure entre autres la tension, le courant et la température. Cela permet de savoir si les cellules opèrent à l'intérieur des caractéristiques fournies par le manufacturier des cellules. Il s'assure également que les modules soient balancés pour éviter une surtension des modules lors de la recharge. Il commande aussi les périphériques auxiliaires, tels que les ventilateurs de refroidissement.
					
		\paragraph{}
		\textbf{Système de gestion:} Il est nécessaire de faire une bonne gestion pour s'assurer du bon fonctionnement de la batterie. Le système de gestion calcul donc l'état de charge, l'état de santé et les limites de puissances. Le système communique ces informations à d'autres systèmes avec un protocole de communication. Il gère également les contacteurs connectant la batterie avec la charge et la recharge.
			
	\subsection{Projet spécial}
		Le cours ELE791 permet aux étudiants de réaliser un projet d'initiation à la recherche ou un projet destiné à un club étudiant participant aux diverses compétitions d'ingénierie. C'est pourquoi une équipe de trois étudiants en génie électrique de l'École de technologie supérieure feront la conception d'un système de protection et de gestion de batterie. Un rapport technique et une présentation orale doivent être effectués au terme de ce projet. Or, l'équipe fait partie du club étudiant Éclipse et réalisera le système de protection et de gestion de batterie du dixième prototype de ce club.
	
		
	\subsection{Club étudiant Éclipse}
		
		\paragraph{}
		Éclipse est un club étudiant composé d'une quarantaine d'étudiants en ingénierie qui ont pour objectif de construire un véhicule alimenter par l'énergie solaire à l'aide de panneaux photovoltaïques. Depuis sa fondation en 1992, neuf prototypes ont été construit en y intégrant les technologies de pointe disponibles au moment de leurs conceptions. Similaire aux voitures électriques, le véhicule solaire possède une chaîne de traction électrique comme moyen de propulsion. Les moteurs roues sont alimentés par des entraînements électriques qui sont à leur tour alimentés par une batterie. Depuis quelques années, l'équipe s'est tournée vers l'utilisation d'accumulateurs au Li-ion. Il est donc nécessaire d'avoir un système de protection et de gestion de batterie.
		
		\paragraph{}
		Une des philosophies du club est de concevoir par les étudiants le plus de modules possible. Le temps de développement du dernier système de protection et de gestion de batterie a pris environ 7 ans et comme la conception d'un nouveau véhicule se fait tous les deux ans, la disposition des batteries change beaucoup. C'est pourquoi les deux derniers prototypes ont utilisé des systèmes acheté de compagnies externes. L'intégration de système provenant de compagnies externes accélère le temps de fabrications du véhicule, mais ne permet pas le contrôle total du système. L'expertise de conception de ces systèmes se perd égalemnt au sein du club puisqu'il y a un gros roulement d'étudiants. Il faut également des systèmes fiables et robustes, ce qui peut être difficile à concevoir lorsque ce système est aussi critique qu'un système de protection et de gestion de batterie. Une défaillance de ce système pourrait mener la destruction partielle ou totale du véhicule et la vie du pilote est en jeu. D'autant plus que le véhicule solaire sera soumis à plusieurs épreuves dans des conditions de route réelles et en circuit fermé durant les compétitions.
	
			
	\subsection{Compétitions}
		Les différentes compétitions dont participe Éclipse ont tous leurs particularités et leurs règlements spécifiques. Il est donc important de concevoir le système de protection et de contrôle de batterie pour qu'il soit compatible avec les différentes compétitions. Il est également important de corriger les défis rencontrés durant les compétitions passées afin de faciliter les compétitions suivantes. Éclipse prévoit participé à la Formula Sun Grand Prix (FSGP) et à l'American Solar Challenge (ASC), deux compétitions se déroulant aux États-Unis. Le club pourra participer à la World Solar Challenge (WSC), compétition se déroulant dans le désert de l'Australie, s'il se classe bien aux deux compétitions précédentes.

		
	\subsubsection{ASC2016}
		
		\paragraph{}
		La dernière compétition était la American Solar Challenge 2016. Cette compétition survient tous les deux ans et elle est séparée en deux étapes. La première étape, la Formula Sun Grand Prix 2016, est une course en circuit fermé. Le club a trois jours pour passer tous les tests techniques pour pouvoir embarquer sur la piste les trois jours suivants. Durant la dernière compétition, il a été très difficile de passer les tests électriques concernant le système de protection de batterie. Puisque le club utilisait un système provient d'une compagnie externe, il était ardu de pouvoir effectuer les tests puisque le club n'avait pas accès à tous les systèmes de protection.
		
		\paragraph{}
		Durant la deuxième partie de la compétition, les différentes équipes participantes devaient parcourir un trajet de 3000 km à travers les États-Unis. Pour cette course d'endurance, il est primordial de bien faire la gestion de l'énergie du véhicule. Cependant, il était très difficile de recueillir l'état de charge de la batterie avec le système actuel. De plus, il ne prenait pas en compte la situation où la batterie est complètement chargée et un freinage regénératif venait recharger la batterie. Le système coupait automatiquement l'alimentation du véhicule, perdant du temps précieux.

		
	\subsubsection{Compétitions futures}
		Il est important de se pencher sur les règlements des prochaines compétitions pour que le système soit conforme et pour que le club ne subisse pas de pénalité.
		
		\paragraph{}
		\textbf{FSGP2017:} Les règlements pour cette compétition sont les plus sévères puisque les équipes doivent passer les tests de vérifications. Ces tests comprennent la détection d'un dépassement de la tension au-dessus et en dessous des spécifications du manufacturier. Le système de protection doit aussi détecter lorsque le courant dépasse la limite permise. De plus, le système doit détecter lorsque la température atteint un niveau trop élevé. Les procédures de ce test sont disponibles dans le document Battery Protection System Test Procedure, document fourni par les responsables de la compétition. Également, l'article 5.4 Protection Circuitry, le système de protection doit être actif lorsque la chimie des cellules utilise du Li-Ion. C'est-à-dire qu'il doit actionner le relais principal lorsque les fautes mentionnées plus haut sont détectées. Le conducteur doit également en être avisé sur son tableau de bord. \cite{reg_fsgp2017}
		
		\paragraph{}
		\textbf{ASC2018:} Les règlements sont très similaires à la FSGP2017. L'article 8.3 de la règlementation détails les requis pour le système de protection de batterie. \cite{reg_asc2018}
		
		\paragraph{}
		\textbf{WSC2019:} Le seul règlement qui concerne le système de protection de batterie est l'article 2.5.8 qui spécifie que les cellules ne doivent pas, en aucun cas, opérer à l'extérieur des tensions, courant et température spécifiés par le manufacturier. \cite{reg_wsc2017}

	
\section{Objectifs}
\paragraph{}
Réaliser tout les objectifs pour avoir un système de protection et de gestion de batterie  n'est malheureusement pas réalisable dans le cadre du cours de projet spécial. Puisque la compétition aura lieu à l'été 2018, l'équipe d'éclipse pourra continuer le projet tout en ce référant à ce cahier des charges. Les objectifs sont donc séparé en deux sections: ELE-791, compétition ASC 2018.  

	\subsection{Objectifs pour ELE-791}
	\paragraph{}	
	Les objectifs pour le cours ce concentrerons sur le minimum requis pour que le système soit utilisable en compétition. Les règlements de la compétition (American Solar Challenge) imposent plusieurs critères pour qu'il puisse être utilisé. Chaque critère est testé pendant la période de vérification technique qui a une plage horaire donnée. Certaines fonctions propre au système de gestion sont aussi nécessaire pour avoir des performances intéressante.
		
		\subsubsection{Protection des modules} \label{protection_module}
		\paragraph{}
		La protection des modules consiste à détecter une surcharge, une décharge excessive, un courant trop élever et une température trop élever. Suite à la détection d'une faute, le système de protection doit amener la batterie à un état sécuritaire indépendamment du conducteur. La réglementation exige que des indicateurs de faute soit installé à l'extérieur du véhicule et sur le tableau de bord. La faute doit être verrouillé et cette dernière doit seulement être effacé manuellement lorsque le véhicule n'est pas en mouvement et que la faute n'est plus présente.
	
		\subsubsection{Balancement des modules}
		\paragraph{}
		Le balancement des cellules signifie que tout les modules sont à la même tension lors de la fin de la charge. Il existe différente stratégie de balancement des modules qui varie en complexité. Bien que ce ne soit pas nécessaire pour que le système ce qualifie, l'étape de balancement à la fin de la charge de la batterie permet d'atteindre un état de charge de 100\% pour tout les modules. Cette étape est critique pour une bonne performance à la compétition. 
				
		\subsubsection{Compatibilité avec le BMS présentement utilisé}
		\paragraph{}
		Éclipse utilise présentement un système de protection et de gestion de batterie avec lequel tout les autres circuits sont compatible et fonctionnel. Pour faciliter l'intégration du projet réalisé dans ce cours, il devra être compatible autant au niveau matériel que des fonctionnalités logiciels avec celui présentement utilisé.		
		
		\subsubsection{Facilité les manipulations lors des vérifications techniques}
		\paragraph{}
		Il est important que les manipulations prennent le moins de temps possible lors des vérification technique à la compétition afin d'avoir une marge de manœuvre pour faire des ajustements ou des réparations de dernière minutes. Les méthodes et procédure de vérification de l'American Solar Challenge sont disponibles et devront être effectué sur le projet avec moins de manipulation que le système utilisé présentement.
		
	\subsection{Objectifs pour la compétition ASC 2018}
	\paragraph{}
	Plusieurs fonctionnalité intéressante pourrait permettre au système de gestion de batterie d'être beaucoup plus performant. Ces fonctionnalité seront développé au cours de l'année lorsque le cours d'ELE-791 sera terminé. 
	
	\subsubsection{Contrôle du système de refroidissement}
	\paragraph{}
	Les modules ne doivent pas atteindre une certaine température spécifiée dans la fiche technique du manufacturier. Un système de refroidissement sera intégré et contrôlé par le système de gestion de batterie. 
	
	\subsubsection{Calcul de l'état de charge}
	\paragraph{}
	Le calcul de l'état de charge est très important au cours de la compétition pour élaborer la stratégie de course. Il existe plusieurs méthodes pour estimer la charge et ces techniques varie en complexité et en précision. Il faudra donc évaluer et tester chaque méthode pour déterminer celle qui répond le mieux aux besoins et l'implémenter. 
	
	\subsubsection{Limites dynamique}
	\paragraph{}
	Afin d'avoir un maximum de performance et de robustesse, des limites dynamiques doivent être communiqué à la charge afin de prévenir les différentes fautes décrite au point \ref{protection_module}. Ces limites sont calculés en temps réel par le système de gestion de batterie.
	
	\subsubsection{Surveiller la batterie et envoyer les informations sur un serveur}
	\paragraph{}
	La batterie d'Éclipse passe la majorité de sa vie utile entreposé à l'atelier sans être surveillé. Une faute ou un état de charge trop bas peut endommager la batterie de façon permanente. Le système de protection et de gestion de la batterie pourrait être connecté sur internet et transmettre les données tout au long de la période d'entreposage afin qu'Éclipse puisse prendre action au besoin.  	



\end{document}


