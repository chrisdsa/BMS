
\section{Objectifs}
Réaliser tout les objectifs pour avoir un système de protection et de gestion de batterie  n'est malheureusement pas réalisable dans le cadre du cours de projet spécial. Puisque la compétition aura lieu à l'été 2018, l'équipe d'éclipse pourra continuer le projet tout en ce référant à ce cahier des charges. Les objectifs sont donc séparé en deux sections: ELE-791, compétition ASC 2018.  

	\subsection{Objectifs pour ELE-791}	
	Les objectifs pour le cours ce concentrerons sur le minimum requis pour que le système soit utilisable en compétition. Les règlements de la compétition (American Solar Challenge) imposent plusieurs critères pour qu'il puisse être utilisé. Chaque critère est testé pendant la période de vérification technique qui a une plage horaire donnée. Certaines fonctions propre au système de gestion sont aussi nécessaire pour avoir des performances intéressante.
	
		\subsubsection{Protection des modules}
		La protection des modules consiste à détecter une surcharge, une décharge excessive, un courant trop élever et une température trop élever. Suite à la détection d'une faute, le système de protection doit amener la batterie à un état sécuritaire indépendamment du conducteur. La réglementation exige que des indicateurs de faute soit installé à l'extérieur du véhicule et sur le tableau de bord. La faute dit être verrouillé et cette dernière doit seulement être effacé manuellement lorsque le véhicule n'est pas en mouvement et que la faute n'est plus présente.
	
		\subsubsection{Balancement des modules}
		Il existe différente stratégie de balancement des modules qui varie en complexité. Bien que ce ne soit pas nécessaire pour que le système ce qualifie, l'étape de balancement à la fin de la charge de la batterie permet d'atteindre un état de charge de 100\% pour tout les modules. Cette étape est critique pour une bonne performance à la compétition. La stratégie de balancement passif en fin de charge est la plus facile et plus rapide à implémenter tout en obtenant les performances désirer. 
				
		\subsubsection{Compatibilité avec le BMS présentement utilisé}
		Éclipse utilise présentement un système de protection et de gestion de batterie avec lequel tout les autres circuits sont compatible et fonctionnel. Pour faciliter l'intégration du projet réalisé dans ce cours, il devra être compatible autant au niveau matériel que des fonctionnalités logiciels avec celui présentement utilisé.		
		
		\subsubsection{Facilité les manipulations lors des vérifications techniques}
		Il est important de prendre le moins de temps possible lors des vérification technique à la compétition afin d'avoir une marge de manœuvre pour faire des ajustements ou des réparations de dernière minutes. Les méthodes et procédure de vérification de l'"American Solar Challenge" sont disponibles et devront être facile à effectuer sur le projet.