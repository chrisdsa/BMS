\section{Ressources}

	
	\subsection{Ressources humaines}
	
		\paragraph{Membres de l'équipe:}
		L'équipe se compose de trois membres en génie électrique chacun possédant des connaissances en conception de carte électronique et en programmation embarquée. Les membres de l'équipe consacrent un minimum de 15 heures par semaine, soit environ 45 heures. Ces heures augmentent durant les parties critiques du projet ou s'il y a des retards sur l'échéancier.
	
		\paragraph{Membres d'Éclipse:}
		Il est possible que certaines parties moins critiques du projet soient déléguées à d'autres membres du club. Par exemple, certains membres ont portés leur intérêt de faire de la recherche ou simplement souder les cartes électroniques. De plus, puisque le système de protection et de gestion de batterie touche d'autres modules du véhicule, les responsables de ces projets peuvent aider l'équipe dans l'intégration du système.
		
		\paragraph{Autres ressources:}
		Les anciens membres ayant travaillés sur des projets similaires peuvent donner à l'équipe de judicieux conseils. Les professeurs de l'école peuvent également aider sur des questions plus techniques ou sur les meilleures méthodes de travail à utiliser. Par exemple, l'équipe peut compter sur l'aide de Dominic Deslandes, professeur responsable de ce projet, et sur Handy Blanchet-Fortin, professeur en électronique de puissance. Puis, le concepteur du système de protection de batterie du club Walking Machine et chargé de laboratoire en ELE542, Shan Meunier, à proposé de répondres aux questions de l'équipe.
		
		
	
	\subsection{Ressources logiciels}
	Pour ce projet, l'équipe utilisera pleusieurs logiciels différents pour concevoir les cartes imprimées (KiCad), écrire le code de la platforme embarqué (Eclipse IDE), simuler le système (Octave), rédiger des textes (LaTeX), gérer les versions (Git) et dessiner les boitiers (Openscad). Ces logiciels sont tous libres, accessibles et ils s'intègrent bien à la méthodologie de l'équipe. Tous les fichiers sont lisibles avec n'importe quel éditeur de texte, ce qui rend le versionnage et le travail collaboratif beaucoup plus facile.\\

	
	\subsection{Ressources matériels}
	L'équipe pourra bénéficier du matériel disponible au Club étudiant Éclipse. Elle a donc accès à de l'équipement de soudure, à du matériel électronique et des instruments de tests. Le magasin électrique de l'école peut également prêter des instruments de mesure et peut fournir des pièces électroniques communes. Les membres de l'équipe possèdent aussi de l'équipement pour réaliser le projet.


	\subsection{Ressources financières}

		\paragraph{Budget}
		Le budget alloué à ce projet est d'environ 2000 dollars, montant versé par Éclipse. La majorité des coûts seront principalement reliés à l'achat de pièces électroniques. Puisque les logiciels utilisés sont libres, l'équipe réduit énormément ses coûts de développement. De plus, la plupart des instruments et outils sont disponibles par l'entremise d'Éclipse ou fournis par les membres de l'équipe.
		
		\paragraph{Commandites}
 		Grâce à une commandite de Labo Circuit, obtenue par Éclipse, l'équipe sauve de l'argent sur la fabrication des cartes électroniques (PCB). Également, une autre commandite de Wurth Electronik réduit le prix des pièces électroniques.