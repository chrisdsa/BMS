
\section{Description fonctionnelle}

	\subsection{Fonctionnalités pour le projet spécial}
	
		\paragraph{Lecture de tension des modules :} 
		Lecture de la tension aux bornes de chaque module.
		
		\paragraph{Lecture du courant de la batterie :}
		Lecture du courant bidirectionnel de la batterie.
		
		\paragraph{Lecture de température :}
		Lecture de la température ambiante à l'intérieur de la batterie et des modules. 
		
		\paragraph{Détection de sur-tension :}
		Une faute de sur-tension est déclenchée lorsque la tension d'un module est plus élevée que celle donné par le manufacturier avec une marge de sécurité.  
		
		\paragraph{Détection de sous-tension :}
		Une faute de sous-tension est déclenchée lorsque la tension d'un module est plus basse que celle donné par le manufacturier avec une marge de sécurité.
		
		\paragraph{Détection de courant trop élevé :}
		Une faute de courant trop élevé est déclenchée lorsque le courant de charge ou décharge ne respecte pas les maximums donnés par le manufacturier avec une marge de sécurité.
		
		\paragraph{Détection de température trop élevé :}
		Une faute de température trop élevé est déclenchée lorsque la température du module est plus haute ou plus basse que la région d'opération donnée par le manufacturier avec une marge de sécurité.
					
		\paragraph{Balancement des modules :}
		La tension de chaque module est amené au même voltage et/ou le même état de charge.
		
		\paragraph{Contrôle des contacteurs :}
		Le système de protection et de gestion de batterie contrôle l'état des contacteurs pour connecter ou déconnecter la batterie. L'état des contacteurs est surveillé pour détecter des rebondissements ou une défaillance.
		
		\paragraph{Précharge :}
		Limitation du courant lorsque la batterie est connecté à la voiture pour charger les charges capacitives.
		
		\paragraph{Communication :}
		Les différentes informations sont communiqués au circuit de contrôle de la voiture solaire et tout autre circuit externe à la voiture.	
		
		\paragraph{Sécuriser la batterie :}
		Lorsqu'une faute est déclenchée, le système de protection amène la batterie à un état sécuritaire. 
			
	
	\subsection{Fonctionnalités pour la compétition}
		

			\paragraph{Détection d'un débalancement des modules :}
			Une alarme est déclenché lorsque l'écart maximum entre deux modules est atteint. Cette alarme indique à l'équipe qu'un module devrait être changé car il diminue les performances de la batterie.  
			
			\paragraph{Gestion du système de refroidissement :} 
			Le système de gestion contrôle les fans pour maintenir une température sécuritaire et optimal si possible.
			
			\paragraph{Estimation de l'état de charge :}
			L'état de la charge de la batterie est estimé en temps réel et communiqué à la télémétrie.
			
			\paragraph{Limites dynamique :}
			Le système de gestion communique les limites de performance (courant et/ou puissance) en temps réel à la voiture afin d’évité que la batterie entre en faute.
			
			\paragraph{Surveillance à distance :}
			Le système de gestion prend quelques mesures à chaque jour lorsque la batterie est entreposée et communique ces mesures à un serveur.		
		
