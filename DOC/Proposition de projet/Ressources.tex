\section{Ressources}

\subsection{Les équipements disponibles}
L'équipe pourra bénéficier du matériel disponible au Club étudiant Éclipse. Elle aura donc accès à de l'équipement de soudure, à du matériel électronique et des instruments de tests. Le magasin électrique de l'école peut également prêter des instruments de mesure et peut fournir des pièces électroniques communes. Les membres de l'équipe possèdent également de l'équipement pour réaliser le projet.

\subsection{Budget}
Le budget alloué à ce projet est d'environ 2000 dollars, montant versé par Éclipse. La majorité des coûts seront principalement reliés à l'achat de pièces électroniques. L'équipe pourra réduire ses coûts puisque les logiciels utilisés sont libres et la plupart des instruments et outils seront disponibles par l'entremise d'Éclipse ou fournis par les membres de l'équipe. Également, une commandite de Labo Circuit, obtenue par Éclipse, permettra à l'équipe de sauver de l'argent sur la fabrication des cartes électroniques (PCB). Également, une autre commandite de Wurth Electronik réduira le prix des pièces électroniques.

\subsection{Échéanciers}
L'équipe utilisera l'échéancier prévu pour un projet de fin d'études puisque les objectifs sont très similaires. Il y aura donc une remise du rapport d'étape vers la mi-session et un rapport final au terme de la session. L'avancement du projet se fera de façon itérative, commençant par une définition des requis puis d'une veille technologique dans les premières semaines. Puis, l'équipe fera la sélection des concepts les plus prometteurs pour ensuite les tester en laboratoires chacune des solutions retenues. Ensuite, l'équipe se divisera les tâches pour effectuer la conception du système de protection de batteries dans la deuxième moitié de la session. Si le temps le permet, l'équipe tentera de fabriquer le BMS pour faire des tests en laboratoire et sur la voiture solaire. Le projet sera également présenté à la fin du projet.
