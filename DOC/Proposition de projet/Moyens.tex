\section{Moyens nécessaires}

\subsection{Logiciels}
Pour ce projet, nous aurons besoin de plusieurs logiciels différent pour concevoir les cartes imprimées, écrire le code de la platforme embarqué, simuler le système, rédiger des texte, gérer les version et dessiner les boitiers. Ces logiciels sont tous libres, accessibles et ils s'intègrent bien à notre méthodologie. Tous les fichiers sont lisible avec n'importe quel éditeur de texte, ce qui rend le versionnage et le travail collaboratif beaucoup plus facile.\\

\subsubsection{KiCad}
KiCad est un logiciel qui gagne en popularité avec "Open Hardware initiative" lancé par le CERN en 2011. Nous utiliserons KiCad pour les schémas des circuits, dessiner les carte électronique et optenir
leur rendu 3D qui pourront être utilisé par les membres de mécaniques.

\subsubsection{Eclipse IDE}


\subsubsection{Octave}


\subsubsection{LaTeX}


\subsubsection{Git}


\subsubsection{OpenSCAD}



\subsection{Plateforme embarqué}
Rtos => nOS, Micrium uC-OSIII

\subsection{Prototypage}

\subsection{}
\subsection{}
\subsection{}