\section{Moyens}

\subsection{Logiciels}
Pour ce projet, l'équipe aura besoin de plusieurs logiciels différent pour concevoir les cartes imprimées (KiCad), écrire le code de la platforme embarqué (Eclipse IDE), simuler le système (Octave), rédiger des texte (LaTeX), gérer les version (Git) et dessiner les boitiers (Openscad). Ces logiciels sont tous libres, accessibles et ils s'intègrent bien à la méthodologie de l'équipe. Tous les fichiers sont lisible avec n'importe quel éditeur de texte, ce qui rend le versionnage et le travail collaboratif beaucoup plus facile.\\

\subsection{Les équipements disponibles}
L'équipe pourra bénéficier du matériel disponible au Club étudiant Éclipse. Elle aura donc accès à de l'équipement de soudure, à du matériel électronique et des instruments de tests. Le magasin électrique de l'école peut également prêter des instruments de mesure et peut fournir des pièces électroniques communes. Les membres de l'équipe possèdent également de l'équipement pour réaliser le projet.

\subsection{Budget}
Le budget alloué à ce projet est d'environ 2000 dollars, montant versé par Éclipse. La majorité des coûts seront principalement reliés à l'achat de pièces électroniques. L'équipe pourra réduire ses coûts puisque les logiciels utilisés sont libres et la plupart des instruments et outils seront disponibles par l'entremise d'Éclipse ou fournis par les membres de l'équipe. Également, une commandite de Labo Circuit, obtenue par Éclipse, permettra à l'équipe de sauver de l'argent sur la fabrication des cartes électroniques (PCB). Également, une autre commandite de Wurth Electronik réduira le prix des pièces électroniques.
