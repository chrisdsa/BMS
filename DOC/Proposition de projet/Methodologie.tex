\section{Méthodologie}
L'équipe va suivre la philosophie du logiciel libre et gérer le projet comme un projet collaboratif. Ces décisions 
ont un impact majeur sur les outils utilisés et dicterons la méthode de travail. \\
La méthodologie du projet n'est pas différente de celle apprise dans le cours de méthodologie (ELE400). Les grandes étapes du projet seront : 
\begin{enumerate}
	\item Identifier les besoins et lire les exigences de la réglementation de la ASC2018.
	\item Rédiger le cahier des charges.
	\item Concevoir, simuler et valider les concepts.
	\item Réaliser un prototype
\end{enumerate}

\subsection{Gestion de projet}
Comme plusieurs projets collaboratifs, GitHub sera utilisé pour entreposer les données et gérer le projet. L'équipe utilisera les "Issues" pour les tâches, la section projet pour suivre leur état et le wiki pour la documentation comme l'échéancier. Contrairement aux logiciels traditionnels, la gestion est transparente et tous les membres de l'équipe peuvent l'éditer simultanément en ligne puisqu'ils en auront l'autorisation. 

\subsection{Échéanciers}
L'équipe utilisera l'échéancier prévu pour un projet de fin d'études puisque les objectifs sont très similaires. Il y aura donc une remise du rapport d'étape vers la mi-session et un rapport final au terme de la session. L'avancement du projet se fera de façon itérative, commençant par une définition des requis puis d'une veille technologique dans les premières semaines. Puis, l'équipe fera la sélection des concepts les plus prometteurs pour ensuite les tester en laboratoires. Ensuite, l'équipe se divisera les tâches pour effectuer la conception du système de protection de batteries dans la deuxième moitié de la session. Si le temps le permet, l'équipe tentera de fabriquer le BMS pour faire des tests en laboratoire et sur la voiture solaire. Le projet sera également présenté sous forme de présentation orale à la fin du projet.

\subsection{Identification des besoins}
Il faudra avant tout faire une recherche sur l’ancien BMS d’Éclipse et d’en retirer les informations importantes. Les anciens du club Éclipse sont également une bonne ressource à ce sujet. La prochaine étape sera de trouver les systèmes de protection de batteries qui sont présentement sur le marché pour voir les fonctionnalités que ceux-ci proposent. Ensuite, il faudra recueillir de la documentation sur les fonctionnalités des BMS à partir de sources académiques et d’en faire une synthèse. Le projet sera découpé en fonction des informations obtenues. 

\subsection{Cahier des charges}
Une fois la synthèse complétée, il faudra déterminer différents choix technologiques pour les différents modules du BMS. Ces choix devront ensuite être évalués par une matrice de décision qui prendra en compte des critères adaptés à un club étudiant, tels que les coûts et les contraintes de temps. Des tests préliminaires seront ensuite effectués pour valider certains concepts.

\subsection{Conception}
Une fois les solutions technologiques choisies, la conception pourra débuter. Il faudra d’abord simuler certaines parties du projet avec des logiciels de simulation, telles qu’Octave. La schématisation se fera ensuite avec des logiciels appropriés tel qu’avec KiCad. Ce projet nécessite également de faire de la programmation. La plateforme de programmation STM32 nucléo sera utilisée puisque le club a récemment migré vers cette plateforme de programmation.

\subsection{Prototype}
Il est important de tester le BMS puisque les piles au Lithium-Ion sont très dangereuses. Lors des premiers tests, nous allons utiliser des piles NiMh puisqu’elles sont moins dangereuses. Les tests pourront également être effectués avec les équipements d’Éclipse et avec l’instrumentation des étudiants de ce projet.
