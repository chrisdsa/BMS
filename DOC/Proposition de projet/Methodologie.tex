\section{Méthodologie}
L'équipe va suivre la philosophie du logiciel libre et gérer le projet comme un projet collaboratif.Ces décicisions 
ont un impact majeur sur la méthodologie et les outils utilisés décrit ci-dessous. \\
Cette méthode de travail n'est pas différente de celle apprise dans le cours de méthodologie (ELE400). Les grandes étape du projet seront : \\
\begin{enumerate}
	\item Identifier les besoins et lire les exigences de la réglementation de la ASC2018.
	\item Rédiger le cahier des charges.
	\item Concevoir, simuler et valider les concepts.
	\item Réaliser un prototype
\end{enumerate}

\subsection{Logiciels}
Pour ce projet, nous aurons besoin de plusieurs logiciels différent pour concevoir les cartes imprimées, écrire le code de la platforme embarqué, simuler le système, le traitement de texte, gérer les version et dessinerles boitiers. c'est logiciels sont tous libres, accessible et ils s'intègrent bien à notre méthodologie. Tous les fichiers sont lisible avec n'importe quel éditeur de texte, ce qui rend le versionnage et le travail collaboratif beaucoup plus facile.\\

\subsubsection{KiCad}
KiCad est un logiciel qui gagne en popularité avec "Open Hardware initiative" lancé par le CERN en 2011. Nous utiliserons KiCad pour les schémas des circuits, dessiner les carte électronique et optenir
leur rendu 3D qui pourront être utilisé par les membres de mécaniques.

\subsubsection{Eclipse IDE}


\subsubsection{Octave}


\subsubsection{LaTeX}


\subsubsection{Git}


\subsubsection{OpenSCAD}



\subsection{Plateforme embarqué}
Rtos => nOS, Micrium uC-OSIII
\subsection{Gestion de projet}
Comme plusieurs projets collaboratif, nous utiliserons GitHub pour entreposer nos données et gérer le projet. Nous utiliserons les "Issues" pour les tâches, la section projet pour suivre leurs état et le wiki pour la documentation comme l'échéancier. Contrairement aux logiciels traditionnel, la gestion est transparente et tout les membres de l'équipe peuvent l'éditer simultanément en linge puisqu'ils en auront bien sûr l'autorisation. 
\subsection{Prototypage}

\subsection{}
\subsection{}
\subsection{}
