\section{Méthodologie}
L'équipe va suivre la philosophie du logiciel libre et gérer le projet comme un projet collaboratif.Ces décicisions 
ont un impact majeur sur les outils utilisés et dictera notre méthode de travail. \\
La méthodologie du projet n'est pas différente de celle apprise dans le cours de méthodologie (ELE400). Les grandes étape du projet seront : 
\begin{enumerate}
	\item Identifier les besoins et lire les exigences de la réglementation de la ASC2018.
	\item Rédiger le cahier des charges.
	\item Concevoir, simuler et valider les concepts.
	\item Réaliser un prototype
\end{enumerate}

\subsection{Gestion de projet}
Comme plusieurs projets collaboratif, nous utiliserons GitHub pour entreposer nos données et gérer le projet. Nous utiliserons les "Issues" pour les tâches, la section projet pour suivre leurs état et le wiki pour la documentation comme l'échéancier. Contrairement aux logiciels traditionnel, la gestion est transparente et tout les membres de l'équipe peuvent l'éditer simultanément en linge puisqu'ils en auront bien sûr l'autorisation. 

