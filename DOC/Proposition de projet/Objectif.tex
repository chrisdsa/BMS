\section{Objectif}
Durant sa compétition à la ASC2016, le club de la voiture solaire Éclipse a eu des difficultés à passer les tests électriques pour se qualifier à la compétition. La raison principale provient du fait que le système de protection de batterie avait été acheté et qu’il ne permettait pas d’effectuer les tests de protections facilement.\\

Il est donc pertinent de concevoir un système de protection de batterie (BMS) au Lithium-ion pour la voiture solaire Éclipse X. Ce projet permettra à Éclipse d’avoir plus de contrôle sur son BMS et donnera l’occasion aux membres du club d’acquérir des connaissances sur le sujet. Ce serait également une bonne façon de moderniser et d'innover cet aspect du véhicule. Ce projet est aussi en accord avec la philosophie d’Éclipse de concevoir tous les modules électriques de la voiture par des étudiants. De plus, le BMS devra répondre à la réglementation de la prochaine compétition, la ASC2018 et devra aussi être compatible avec le présent BMS. Il faudra donc tenir compte des différentes fonctionnalités et des connexions déjà existantes pour qu’il soit interchangeable.
