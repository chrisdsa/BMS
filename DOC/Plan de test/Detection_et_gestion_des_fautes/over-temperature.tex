
\section{Détection et gestion d'une température trop élevée}

% Objectif et critère de réussite
\begin{description}
	\item[Objectif :] Détecter la température trop élevée de la batterie.
	\item[Critère de réussite :] Détecter la température trop élevée dans un délai de $200\text{ms}$.
\end{description}

% Équipement nécessaire
\subsection*{Équipements nécessaires}
\begin{enumerate}
	\item Source de chaleur (heat gun, séchoir à cheveux);
	\item Multimètre capable de mesurer la température.
\end{enumerate}	

% Procédure de test
\subsection*{Procédure de test}
\begin{enumerate}
	\item Fixer les 2 capteurs de température ensemble le plus près possible; 
	\item Connecter une source de tension à l'entrée du relais.
	\item Connecter l'oscilloscope et une charge à la sortie du relais.
	\item Chauffer les capteurs de température et noter la valeur du multimètre et celle donnée par le système de protection de batterie lorsque la batterie est sécurisée.
	\item Noter le temps que le système a pris pour sécuriser la batterie.
\end{enumerate}