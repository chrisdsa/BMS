\documentclass[12pt,letterpaper]{article}
\usepackage{hyperref}
\hypersetup{colorlinks=true,allcolors=black}
\usepackage[utf8]{inputenc}
\usepackage[frenchb]{babel}
\usepackage[T1]{fontenc}
\usepackage[margin=0.75in]{geometry}
\usepackage{hhline}
\usepackage{amsmath}
\usepackage{pgfplots}
\pgfplotsset{compat=newest}
\usepackage{multicol}
\usepackage{graphicx}
\graphicspath{}
\hyphenpenalty=100
\usepackage{caption}
\usepackage{enumitem}
\usepackage{textcomp}
\usepackage{float}
\floatstyle{plaintop}
\restylefloat{table}

\begin{document}
	
%====== Page de presentation ======%
	\hypersetup{pageanchor=false}
	%================ Page titre ================

\title{
	\textbf{Plan de test} \\
	\vspace{2cm}
	Système de protection et de gestion de batterie Li-ion	
}
\author{
	Daigneault-St-Arnaud, Christian, DAIC30099006 \\
	Gagnon-Bourassa, Julien, GAGJ23108601 \\
	Cusson-Larocque, Olivier, CUSO09048905	
}
\newcommand{\cours}{ELE791 - Projets spéciaux }
\newcommand{\prof}{Deslandes, Dominic}



\makeatletter
\begin{titlepage}


	\pagenumbering{gobble}
	\centering
	{\Huge \@title}\\ 
	\vspace{3cm}
	{\large Par \\
		\vspace{0.5cm}
		\@author \\
		\vspace{3cm}
		\cours \\
		\vspace{0.5cm}
		\prof \\
		\vspace{3.5cm}
		\@date \\
		\vspace{3.5cm}
		\'{E}COLE DE TECHNOLOGIE SUP\'{E}RIEURE \\
		UNIVERSIT\'{E} DU QUÉBEC
	}
\end{titlepage}
\makeatother





	\newpage
%====== Table des matieres ======%
	\pagenumbering{roman}
	%\tableofcontents
	%\listoftables
	%\listoffigures	
	%\newpage
	

%====================== INCLUSION DES PARTIES ======================
	\pagenumbering{arabic}
	% Test de la polarité inverse du module
	
\section{Validation de la protection de polarité inverse d'un module}

	% Objectif et critère de réussite
	\begin{description}
		\item[Objectif :] Valider la protection à l'entrée du circuit de mesure de la tension d'un module.
		\item[Critère de réussite :] Obtenir une différence de potentiel nul sur l'alimentation du ADC.
	\end{description}
	
	% Équipement nécessaire
	\subsection*{Équipements nécessaire}
	\begin{enumerate}
		\item Source de tension ajustable de 0V à 5V ou plus;
		\item Multimètre avec une précision d'au moins 1mV;
	\end{enumerate}	
	
	% Procédure de test
	\subsection*{Procédure de test}
	\begin{enumerate}
		\item Connecter la source à 4.5V au borne inversé du circuit de lecture de tension d'un module;
		\item Mesurer le potentiel au borne de l'alimentation du ADC.
	\end{enumerate}
	% Lecture de la tension des modules
	
\section{Validation de la résolution de la lecture de tension d'un module}

	% Objectif et critère de réussite
	\begin{description}
		\item[Objectif :] Valider la précision de la lecture de tension sur toute la plage.
		\item[Critère de réussite :] Obtenir une résolution de minimum $\pm 10 \text{mV}$.
	\end{description}

	% Équipement nécessaire
	\subsection*{Équipements nécessaires}
	\begin{enumerate}
		\item Source de tension ajustable de 0V à 5V ou plus;
		\item Multimètre avec une précision d'au moins 1mV;
	\end{enumerate}	

	% Procédure de test
	\subsection*{Procédure de test}
	\begin{enumerate}
		\item Connecter la source à 0V aux bornes du circuit de lecture de tension d'un module;
		\item Faire varier la source de 0V à 4.8V, noter chaque valeur de la source et celle donnée par le système de protection de batterie.
	\end{enumerate}
	% Lecture du courant
	
\section{Validation de la résolution de la lecture de courant de la batterie}

	% Objectif et critère de réussite
	\begin{description}
		\item[Objectif :] Valider la précision de la lecture de courant sur toute la plage.
		\item[Critère de réussite :] Obtenir une résolution de minimum 2\%.
	\end{description}
	
	% Équipement nécessaire
	\subsection*{Équipements nécessaires}
	\begin{enumerate}
		\item Source de tension ajustable;
		\item Multimètre avec une précision d'au moins 1mV;
		\item Diviseur de tension
	\end{enumerate}	
	
	% Procédure de test
	\subsection*{Procédure de test}
	\begin{enumerate}
		\item Remplacer la résistance "shunt" par les deux bornes inférieures du diviseur de tension; 
		\item Connecter la source de tension à 0V au diviseur de tension;
		\item Varier la tension de la source et noter chaque valeur de la source et celle donnée par le système de protection de batterie.
		\item Inverser les bornes de la source.
		\item Répéter le point 3.
	\end{enumerate}
	% Lecture de la température
	
\section{Validation de la résolution de la lecture de température}

	% Objectif et critère de réussite
	\begin{description}
		\item[Objectif :] Valider la précision de la lecture de température sur toute la plage.
		\item[Critère de réussite :] Obtenir une résolution de minimum $2 \degres$.
	\end{description}
	
	% Équipement nécessaire
	\subsection*{Équipements nécessaire}
	\begin{enumerate}
		\item Source de chaleur (heat gun, séchoir à cheveux);
		\item Multimètre capable de mesurer la température.
	\end{enumerate}	
	
	% Procédure de test
	\subsection*{Procédure de test}
	\begin{enumerate}
		\item Fixer les 2 senseurs de température ensemble le plus près possible; 
		\item Chauffer les senseur de température et noter chaque valeur du multimètre et celle donné par le système de protection de batterie.
	\end{enumerate}
	% Détection de surtension
	
\section{Détection et gestion d'une surtension d'un module}

% Objectif et critère de réussite
\begin{description}
	\item[Objectif :] Détecter la surtension du module.
	\item[Critère de réussite :] Détecter la surtension du module dans un délais de $200\text{ms}$.
\end{description}

% Équipement nécessaire
\subsection*{Équipements nécessaire}
\begin{enumerate}
	\item Source de tension ajustable de 0V à 5V ou plus;
	\item Multimètre avec une précision d'au moins 1mV;
	\item Oscilloscope
\end{enumerate}	

% Procédure de test
\subsection*{Procédure de test}
\begin{enumerate}
	\item Connecter la source de tension à 3V et le multimètre aux bornes du circuit de lecture de tension d'un module;
	\item Connecter une source de tension à l'entrée du relais.
	\item Connecter l'oscilloscope et une charge à la sortie du relais.
	\item Faire varier lentement la source de 0V à 4.8V, noter la valeur de la source et celle donné par le système de protection de batterie lorsque la batterie est sécurisé. La batterie est considéré sécurisé lorsque la sortie du relais est à 0V. 
	\item Noter le temps que le système à pris pour sécuriser la batterie.
\end{enumerate}
	% Détection de sous-tension
	
\section{Détection et gestion d'une sous-tension d'un module}

% Objectif et critère de réussite
\begin{description}
	\item[Objectif :] Détecter la sous-tension du module.
	\item[Critère de réussite :] Détecter la sous-tension du module dans un délai de $200\text{ms}$.
\end{description}

% Équipement nécessaire
\subsection*{Équipements nécessaires}
\begin{enumerate}
	\item Source de tension ajustable de 0V à 5V ou plus;
	\item Multimètre avec une précision d'au moins 1mV;
	\item Oscilloscope
\end{enumerate}	

% Procédure de test
\subsection*{Procédure de test}
\begin{enumerate}
	\item Connecter la source de tension à 3V et le multimètre aux bornes du circuit de lecture de tension d'un module;
	\item Connecter une source de tension à l'entrée du relais.
	\item Connecter l'oscilloscope et une charge à la sortie du relais.
	\item Faire varier lentement la source de 3V à 0V, noter la valeur de la source et celle donnée par le système de protection de batterie lorsque la batterie est sécurisée. La batterie est considérée sécurisée lorsque la sortie du relais est à 0V. 
	\item Noter le temps que le système a pris pour sécuriser la batterie.
\end{enumerate}
	% Détection de température trop élevé
	
\section{Détection et gestion d'une température trop élevée}

% Objectif et critère de réussite
\begin{description}
	\item[Objectif :] Détecter la température trop élevée de la batterie.
	\item[Critère de réussite :] Détecter la température trop élevée dans un délai de $200\text{ms}$.
\end{description}

% Équipement nécessaire
\subsection*{Équipements nécessaires}
\begin{enumerate}
	\item Source de chaleur (heat gun, séchoir à cheveux);
	\item Multimètre capable de mesurer la température.
\end{enumerate}	

% Procédure de test
\subsection*{Procédure de test}
\begin{enumerate}
	\item Fixer les 2 capteurs de température ensemble le plus près possible; 
	\item Connecter une source de tension à l'entrée du relais.
	\item Connecter l'oscilloscope et une charge à la sortie du relais.
	\item Chauffer les capteurs de température et noter la valeur du multimètre et celle donnée par le système de protection de batterie lorsque la batterie est sécurisée.
	\item Noter le temps que le système a pris pour sécuriser la batterie.
\end{enumerate}
	% Détection de courant excessif
	
\section{Détection et gestion d'un courant trop élevé}

% Objectif et critère de réussite
\begin{description}
	\item[Objectif :] Détecter le courant trop élevé.
	\item[Critère de réussite :] Détecter le courant trop élevé dans un délais de $200\text{ms}$.
\end{description}

% Équipement nécessaire
\subsection*{Équipements nécessaire}
\begin{enumerate}
	\item Source de tension ajustable;
	\item Multimètre avec une précision d'au moins 1mV;
	\item Diviseur de tension
\end{enumerate}	

% Procédure de test
\subsection*{Procédure de test}
\begin{enumerate}
	\item Remplacer la résistance "shunt" par les deux bornes inférieur du diviseur de tension; 
	\item Connecter la source de tension à 0V au diviseur de tension;
	\item Connecter une source de tension à l'entrée du relais.
	\item Connecter l'oscilloscope et une charge à la sortie du relais.
	\item Varier la tension de la source branché au diviseur de tension. Noter la valeur de la source et celle donné par le système de protection de batterie lorsque la batterie est sécurisé. La batterie est considéré sécurisé lorsque la sortie du relais est à 0V.
	\item Inverser les bornes de la source connecté au diviseur de tension.
	\item Répété le points 5.
\end{enumerate}
	
	


\end{document}


