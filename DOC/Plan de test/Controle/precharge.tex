
\section{Précharge}

% Objectif et critère de réussite
\begin{description}
	\item[Objectif :] Valider la séquence de précharge.
	\item[Critère de réussite :] Le contacteur sur la borne positive de la batterie est enclenché lorsque la tension à ses bornes est nulle, le tout dans un délai ne dépassant pas 10 secondes.
\end{description}

% Équipement nécessaire
\subsection*{Équipements nécessaires}
\begin{enumerate}
	\item Source de tension au voltage de la batterie;
	\item Oscilloscope;
	\item Charge capacitive représentant la capacité de l'ensemble des circuits de la voiture.
\end{enumerate}	

% Procédure de test
\subsection*{Procédure de test}
\begin{enumerate}
	\item Brancher l'entrée des 2 contacteurs sur les deux bornes de la source d'alimentation;
	\item Brancher le circuit de précharge;
	\item Brancher la charge et l'oscilloscope aux sorties des contacteurs et du circuit de précharge;
	\item Brancher l'oscilloscope au signal logique du circuit de précharge et du contacteur sur le positif.
	\item Brancher le système de protection sur les signaux de contrôle des deux contacteurs et du circuit de précharge;
	\item Envoyer un message CAN pour enclencher la précharge;
	\item Noter le résultat; 
\end{enumerate}