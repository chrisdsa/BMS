\section{Rappel des objectifs}

	\paragraph{}
	L’équipe s’est fixé des objectifs aux début du projet qui était réalisable dans le cadre du cours d’ELE791. Les objectifs sont les suivants :

	\subsection{Protection des modules}
	
		\paragraph{}
		La protection des modules inclue la détection de surtension, de sous-tension, de surintensité et de surchauffe. À la suite d’une détection d’erreur, les contacteurs doivent s’ouvrir dans un délai de moins de deux secondes et un témoin lumineux doit avertir l’utilisateur. Le système de protection et de gestion de batterie doit être redémarrer par l’utilisateur et ce s’il n’y a pas d’erreur.

	\subsection{Balancement des modules}

		\paragraph{}
		Comme les modules n’ont pas tous la même capacité, des débalancements peuvent survenir lorsque la batterie est presque pleine ou presque vide. Pour éliminer ces débalancement, il faut saigner (bleeder) les modules, c’est-à-dire de décharger les modules déséquilibrés pour les amener aux mêmes niveaux de charges.
		
	\subsection{Compatibilité avec le BMS Lithium Balance}

		\paragraph{}
		Éclipse utilise présentement le BMS de Lithium Balance. Celui-ci est fonctionnel, mais il n’offre pas la possibilité de faire des modifications. Éventuellement, il sera remplacer par le projet actuel lorsqu'il sera jugé terminé. Par contre, si jamais un problème survient en compétition, les connexions doivent être les mêmes pour facilement remplacer le nouveau BMS par l’ancien. Ce faisant, le projet doit inclure les fonctionnalités offertes par le BMS de Lithium Balance.

	\subsection{Facilitation des manipulations lors des vérifications techniques}

		\paragraph{}
		Une des raisons de fabriquer le système de protection et de gestion de batterie est de facilité les manipulations durant la vérification techniques des compétitions du club Éclipse. Avec le BMS actuel, les manipulations sont assez difficiles à effectuer. Il faut donc, avec le moins de manipulations possibles, aisément passer les tests de la vérifications techniques du BMS.

