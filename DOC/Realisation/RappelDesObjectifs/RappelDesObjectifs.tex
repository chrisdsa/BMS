\section{Rappel des objectifs}

	\paragraph{}
	L’équipe s’est fixé des objectifs aux début du projet qui était réalisable dans le cadre du cours d’ELE791. Les objectifs sont les suivants :

	\subsection{Protection des modules}
	
		\paragraph{}
		La protection des modules inclue la détection de surtension, de sous-tension, de surintensité et de surchauffe. À la suite d’une détection d’erreur, les contacteurs doivent s’ouvrir dans un délai de moins de deux secondes et un témoin lumineux doit avertir l’utilisateur. Le système de protection et de gestion de batterie doit être redémarrer par l’utilisateur et ce s’il n’y a pas d’erreur.

	\subsection{Balancement des modules}

		\paragraph{}
		Comme les modules n’ont pas tous les mêmes capacités, des débalancements peuvent survenir lorsque la batterie est presque pleine ou presque vide. Pour éliminer ces débalancement, il faut saigner (bleeder) les modules, c’est-à-dire de les décharger les modules déséquilibrés pour les amener au même niveau de charge. Pour ce faire, l’équipe a choisi de faire une décharge passive, soit de décharger les modules dans une série de résistance. 

	\subsection{Compatibilité avec le BMS Lithium Balance}

		\paragraph{}
		Éclipse utilise présentement le BMS de Lithium Balance. Celui-ci est fonctionnel, mais il n’offre pas la possibilité de faire des modifications. Le présent projet devra remplacer éventuellement ce BMS. Par contre, si jamais un problème survient en compétition, les connexions doivent être les mêmes pour facilement remplacer le nouveau BMS par l’ancien. Ce faisant, le projet doit inclure les fonctionnalités offertes par le BMS de Lithium Balance.

	\subsection{Facilitation des manipulations lors des vérifications techniques}

		\paragraph{}
		Une des raisons de fabriquer le système de protection et de gestion de batterie est la facilité de manipulation durant les compétitions du club Éclipse. Lors des vérifications techniques, il n’est pas facile de passer les tests avec la solutions présentement utilisé. Il faut donc, avec le moins de manipulations possibles, facilement passer les tests de la vérifications techniques du BMS.

