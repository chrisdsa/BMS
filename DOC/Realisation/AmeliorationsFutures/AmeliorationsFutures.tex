\section{Améliorations futures}

	\paragraph{}
	L'équipe s'est fixé des objectifs qu'ils jugeaient atteignables dans le cadre du cours d'ELE791. Cependant, le projet de protection et de gestion de batterie Li-Ion est loin d'être terminé. Le projet est au stade de preuve de concept puisque plusieurs fonctions n'ont pas encore été implémenté. Les auteurs, conscient que le projet sera pris par d'autres étudiants, tenaient à dresser la liste des éléments à implémenter.

	\subsection{Filtre de détection}

		\paragraph{}
		Bien que la détection de fautes est fonctionnelle pour la protection des modules de la batterie, elle est trop sensible. Une pointe momentanée de tension ou de courant peut rendre le système en erreur, stoppant du même coup le véhicule solaire. Lorsque la batterie est chargée ou lorsque la demande en énergie est élevée, des pointes de tension peuvent survenir. Pour éviter qu'une erreur non désirée survienne, il faut implémenter un filtre logiciel qui acceptera un délai d'une seconde avant de tomber une faute. Si le système retrouve son état normal durant ce délai, l'erreur n'ouvrira pas les contacteurs. Le délai d'une seconde est permis par la réglementation de la compétition.

	\subsection{Fonctions du module maître}

		\paragraph{}
		Plusieurs fonctions du module maître sont fonctionnels mais ne sont pas implémentées dans le code applicatif du projet.

		\paragraph{}
		La précharge est présentement désactivée dans le code applicatif. Pour qu'elle soit fonctionnelle, il faut pouvoir lire la tension du contacteur principal et du contacteur MPPT. On ferme ensuite les relais de précharge qui fera passer un petit courant pour remplir les condensateurs du système et qui fera baisser la différence de potentiel aux bornes des contacteurs. On enclenche ensuite les contacteurs une fois la précharge effectuée. Présentement, puisqu'aucune charge n'est présente, on ferme les trois contacteurs sans passer par la précharge.

		\paragraph{}
		Le module maître possède aussi des fonctions venant du BMS Lithium Balance. Ces fonctions sont tous reliées avec le connecteurs 22 broches, avec le même brochages que le BMS de Lithium Balance. Les circuits sont présents sur la carte électronique mais il ne sont pas pris en charge par le code applicatif. De plus, les trois circuits d'entrée/sortie à utilisation générale, qui serviront à contrôler des ventilateurs, sont à implémenter. Une communication sériel est aussi présente sur le connecteur 22 broches et pourra servir à programmer et déverminer le BMS à distance.

		\paragraph{}
		Certains correctifs doivent être apportés sur la carte électronique de la deuxième version du module maître. Par exemple, les neutres communs devront tous être reliés et les broches de l'amplificateur opérationnel de la partie de lecture de tension des contacteurs doivent être inversées. Également, il a été décidé que les fonctionnalités non utilisées venant du BMS de Lithium Balance seront éliminées. Donc, les deux convertisseurs digitales à analogique, le signal à largeur d'impulsion variable et l'entré analogiques, avec le convertisseur analogique à digital ne seront pas présents dans la prochaine itération de la carte électronique du module maître.

	\subsection{Fonctions du module esclave}

		\paragraph{}
		Présentement, les modules esclaves ont un numéro d'identification (ID) prédéfini lors de la programmation de ceux-ci. Un algorithme doit être développé pour que les modules esclaves se négocie un ID automatiquement au démarrage du BMS. Idéalement, les modules se souviendraient de leur dernier ID et essaieraient de d'avoir le même à chaque démarrage.

	\subsection{Interface graphique}
		\paragraph{}
		Une interface graphique pour le module maître servira à configurer différents paramètres comme les caractéristiques de la batterie ainsi que les limites utilisées par les fonctions de protection. Ceci évitera de devoir reprogrammer les cartes à chaque fois qu'un changement doit être effectué.

		\paragraph{}
		Au moment de l'écriture de ce rapport, une démarche est en cours avec le chargé de cours en génie logiciel et T.I. Alain Dion pour faire développer l'interface par ses étudiants dans le cadre du cours LOG410.

	\subsection{Balancement des modules}

		\paragraph{}
		Le balancement sert à égaliser la tension des modules lors de la recharge. Comme les modules n'ont pas tous exactement la même capacité réelle, certains atteigneront leur tension maximale avant les autres. Le courant de charge devra donc être détourné pour ceux-ci, le temps que les modules plus faibles rattrapent leur écart.

	\subsection{Contrôle du système de refroidissement}

		\paragraph{}
		Le boitier de la batterie sera refroidi par trois ventilateurs contrôlés par le BMS.

	\subsection{Calcul de l’état de charge}

		\paragraph{}
		Le calcul de l'état de charge sert à connaître la capacité restante de la batterie. Cette information est essentielle pour une bonne stratégie de course. Le matériel a été prévu pour implémenter cette fonctionnalité en utilisant, entre autres, la méthode du "couloumb counter".

	\subsection{Utilisation de limites dynamiques}

		\paragraph{}
		Les limites dynamiques servent à limiter le courant d'entrée et de sortie dans la batterie selon son état de charge. Si la batterie est presque pleine et que le véhicule entre en mode regénération, on voudra limiter dynamiquement le courant pour ne pas que la tension de la batterie dépasse le seuil de sécurité. Le même principe s'applique en décharge lorsque la batterie est presque vide. 

	\subsection{Surveillance de la batterie}

		\paragraph{}
		Il serait intéressant d'avoir un système qui puisse continuer de surveiller la batterie lorsque celle-ci est hors du véhicule. Lors de l'utilisation normale de la batterie dans le véhicule, c'est un humain, grâce à la carte de télémétrie qui peut faire la surveillance. Le système à développer remplacerait en quelque sorte la télémétrie du véhicule et pourrait envoyer des notifications par courriel en cas de problèmes. Une plateforme BeagleBone Black ou Raspberry Pie pourrait être utilisé à cette fin.



