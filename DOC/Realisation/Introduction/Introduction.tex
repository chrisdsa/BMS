\section{Introduction}

	\paragraph{}
	Dans le cadre du cours ELE791, les étudiants réalisent un projet destiné à un club étudiant participant aux diverses compétitions d’ingénierie. Les auteurs ont fait la conception d’un système de protection et de gestion de batterie Li-Ion pour le véhicule solaire Éclipse. 
	
	\paragraph{}
	Avant de réaliser ce projet, l’équipe a rédigé un cahier de charges, résumant les besoins du client, les objectifs du projet, les spécifications à atteindre, les ressources disponibles et l’échéancier. Par la suite, le cahier de conception faisait état des choix technologiques sélectionnés par les membres de l’équipe pour répondre aux différents besoins du projet. De plus, l’équipe s’est donnée comme objectif de faire un prototype fonctionnel aux termes du cours. Un cahier de réalisation est donc nécessaire et celui-ci a donc été rédigé durant la réalisation du projet.
	
	\paragraph{}
	Le contenu du cahier de réalisation inclut le rappel des objectifs et les étapes de réalisations du module esclave, du module maître et du module de lecture de courant. De plus, il est question des résultats obtenus en termes d’objectifs accomplis et d’un tableau comparant les performances et caractéristiques du BMS. Puis, l’équipe présente les améliorations futures à apporter au projet qui seront implantées par les prochains étudiants voulant continuer le système de protection et de gestion de batterie Li-Ion.
