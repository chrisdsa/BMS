\section{Introduction}

	\paragraph{}
	Dans le cadre du cours ELE791, les étudiants ont fait la conception d’un système de protection et de gestion de batterie Li-Ion pour le véhicule solaire Éclipse. Avant d’effectuer la réalisation de ce projet, l’équipe rédigé un cahier de charges, résumant les besoins du client, les objectifs du projet, les spécifications à atteindre, les ressources disponibles et l’échéancier. Par la suite, le cahier de conception faisait état des choix technologiques choisis par les membres de l’équipe pour répondre au différents besoins du projet. De plus, l’équipe s’est donné comme objectif de faire un prototype fonctionnel aux termes du cours.
	Ce présent document est donc le cahier de réalisation qui explique les différentes étapes de réalisations, les problèmes rencontrés ainsi que les solutions apportées. Le document défini aussi les performances et les caractéristiques du projet.
	Ce document servira également aux prochains étudiants voulant continuer le projet.
	